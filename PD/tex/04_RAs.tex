\newcommand{\pondRAp}[1]{\pondRA{#1}\%}
\newcommand{\pondRA}[1]{%
  \ifnum#1=1
    10%
  \else\ifnum#1=2
    20%
  \else\ifnum#1=3
    30%
  \else\ifnum#1=4
  	30%
  \else\ifnum#1=5
  	30%
  \else\ifnum#1=6
  	30%
  \else\ifnum#1=7
  	30%
  \else\ifnum#1=8
  	30%
  \else\ifnum#1=9
  	30%
  \fi\fi\fi\fi\fi\fi\fi\fi\fi
}

A continuación, presentamos el elemento vertebrador de la práctica docente de esta programación: los \index{Resultados de Aprendizaje}{Resultados de Aprendizaje (RAs)} definidos en el \lawref{rd:450-2010}. Estos son los aprendizajes que debe adquirir un alumno al terminar el módulo. 

Los RAs han sido diseñados para que sean una guía práctica para desarrollar progresivamente las competencias del módulo, actuando como objetivos concretos y evaluables en la formación del alumnado.
%
Cada competencia queda cubierta y desarrollada a través de uno o varios resultados de aprendizaje, garantizando así una adquisición integral y sistemática de los conocimientos y habilidades definidos en el módulo de Programación.

La \lawref{ley:nueva-fp} establece que debe trabajarse parte del Currículo en las FCTs, otorgando libertad al centro y al departamento para la concreción de cuáles y en qué porcentaje. No obstante, sí existe la recomendación por parte de la Consejería de no dejar un RA completo para las FCTs. Además, el centro ha establecido el criterio de que no más del 10\% del peso de un RA puede corresponder a la FCT. 

En este caso, solamente un RA (el 3) será parcialmente calificado en la FCTs. En el apartado de calificación se concretarán los instrumentos y ponderaciones exactos.

En la siguiente tabla se presentan los RAs indicando, para cada uno, tanto su ponderación respecto de la calificación total como la ponderación de cada \index{Criterios de Evaluación (CE)}{Criterios de Evaluación (CE)} dentro del RA. La suma de las ponderaciones de los CEs suma 100\% para cada RA, así como la ponderación de los RAs suma 100\%.

\newpage
\begin{longtable}{|>{\raggedright\arraybackslash}p{3.5cm}|p{10cm}|>{\centering\arraybackslash}p{1.2cm}|}
		\hline
		\vspace{0.2cm}\textbf{RA} & \vspace{0.2cm}\textbf{Criterio de evaluación} &\vspace{0.2cm} \textbf{$\%$}\\
		% Siguiente fila
		\hline\endhead	
\multirow{9}{*}{\parbox{3cm}{\vspace{0.4cm}\textbf{RA1 (\pondRAp{1})}\label{RA1}:\\ Reconoce la estructura de un programa informático, identificando y relacionando los elementos propios del lenguaje de programación utilizado.}}
		& \label{RA1:CEa}a) Se han identificado los bloques que componen la estructura de un programa informático. 
		&
		%PONDERACIÓN
		 \\
		 \cline{2-3}
		& \label{RA1:CEb}b) Se han creado proyectos de desarrollo de aplicaciones. 
		&
		%PONDERACIÓN
		 \\
		 \cline{2-3}
		& \label{RA1:CEc}c) Se han utilizado entornos integrados de desarrollo. 
		&
		%PONDERACIÓN
		 \\
		 \cline{2-3}
		& \label{RA1:CEd}d) Se han identificado los distintos tipos de variables y la utilidad específica de cada uno. 
		&
		%PONDERACIÓN
		 \\
		 \cline{2-3}
		& \label{RA1:CEe}e) Se ha modificado el código de un programa para crear y utilizar variables. 
		&
		%PONDERACIÓN
		 \\
		 \cline{2-3}
		& \label{RA1:CEf}f) Se han creado y utilizado constantes y literales. 
		&
		%PONDERACIÓN
		 \\
		 \cline{2-3}
		& \label{RA1:CEg}g) Se han clasificado, reconocido y utilizado en expresiones los operadores del lenguaje. 
		&
		%PONDERACIÓN
		 \\
		 \cline{2-3}
		& \label{RA1:CEh}h) Se ha comprobado el funcionamiento de las conversiones de tipo explícitas e implícitas. 
		&
		%PONDERACIÓN
		 \\
		 \cline{2-3}
		& \label{RA1:CEi}i) Se han introducido comentarios en el código. 
		&
		%PONDERACIÓN
		 \\
		 \cline{2-3}
		\hline
		% Siguiente fila
\multirow{9}{*}{\parbox{3cm}{\vspace{0.4cm}\textbf{RA2 (\pondRAp{2})}\label{RA2}:\\ Escribe y prueba programas sencillos, reconociendo y aplicando los fundamentos de la programación orientada a objetos.}}
		& \label{RA2:CEa}a) Se han identificado los fundamentos de la programación orientada a objetos. 
		&
		%PONDERACIÓN
		 \\
		 \cline{2-3}
		& \label{RA2:CEb}b) Se han escrito programas simples. 
		&
		%PONDERACIÓN
		 \\
		 \cline{2-3}
		& \label{RA2:CEc}c) Se han instanciado objetos a partir de clases predefinidas. 
		&
		%PONDERACIÓN
		 \\
		 \cline{2-3}
		& \label{RA2:CEd}d) Se han utilizado métodos y propiedades de los objetos. 
		&
		%PONDERACIÓN
		 \\
		 \cline{2-3}
		& \label{RA2:CEe}e) Se han escrito llamadas a métodos estáticos. 
		&
		%PONDERACIÓN
		 \\
		 \cline{2-3}
		& \label{RA2:CEf}f) Se han utilizado parámetros en la llamada a métodos. 
		&
		%PONDERACIÓN
		 \\
		 \cline{2-3}
		& \label{RA2:CEg}g) Se han incorporado y utilizado librerías de objetos. 
		&
		%PONDERACIÓN
		 \\
		 \cline{2-3}
		& \label{RA2:CEh}h) Se han utilizado constructores. 
		&
		%PONDERACIÓN
		 \\
		 \cline{2-3}
		& \label{RA2:CEi}i) Se ha utilizado el entorno integrado de desarrollo en la creación y compilación de programas simples. 
		&
		%PONDERACIÓN
		 \\
		 \cline{2-3}
		\hline
		% Siguiente fila
\multirow{9}{*}{\parbox{3cm}{\vspace{0.4cm}\textbf{RA3 (\pondRAp{3})}\label{RA3}:\\ Escribe y depura código, analizando y utilizando las estructuras de control del lenguaje.\\ \textit{Este es el RA que será parcialmente evaluado en las FCTs}}}
		& \label{RA3:CEa}a) Se ha escrito y probado código que haga uso de estructuras de selección. 
		&
		5\%
		 \\
		 \cline{2-3}
		& \label{RA3:CEb}b) Se han utilizado estructuras de repetición. 
		&
		5\%
		 \\
		 \cline{2-3}
		& \label{RA3:CEc}c) Se han reconocido las posibilidades de las sentencias de salto. 
		&
		5\%
		 \\
		 \cline{2-3}
		& \label{RA3:CEd}d) Se ha escrito código utilizando control de excepciones. 
		&
		% PONDERACION
		 \\
		 \cline{2-3}
		& \label{RA3:CEe}e) Se han creado programas ejecutables utilizando diferentes estructuras de control. 
		&
		25\%
		 \\
		 \cline{2-3}
		& \label{RA3:CEf}f) Se han probado y depurado los programas. 
		&
		25\%
		 \\
		 \cline{2-3}
		& \label{RA3:CEg}g) Se ha comentado y documentado el código. 
		&
		25\%
		 \\
		 \cline{2-3}
		& \label{RA3:CEh}h) Se han creado excepciones. 
		&
		%PONDERACIÓN
		 \\
		 \cline{2-3}
		& \label{RA3:CEi}i) Se han utilizado aserciones para la detección y corrección de errores durante la fase de desarrollo. 
		&
		%PONDERACIÓN
		 \\
		 \cline{2-3}
		\hline
		% Siguiente fila
\multirow{9}{*}{\parbox{3cm}{\vspace{0.4cm}\textbf{RA4 (\pondRAp{4})}\label{RA4}:\\ Desarrolla programas organizados en clases analizando y aplicando los principios de la programación orientada a objetos.}}
		& \label{RA4:CEa}a) Se ha reconocido la sintaxis, estructura y componentes típicos de una clase. 
		&
		%PONDERACIÓN
		 \\
		 \cline{2-3}
		& \label{RA4:CEb}b) Se han definido clases. 
		&
		%PONDERACIÓN
		 \\
		 \cline{2-3}
		& \label{RA4:CEc}c) Se han definido propiedades y métodos. 
		&
		%PONDERACIÓN
		 \\
		 \cline{2-3}
		& \label{RA4:CEd}d) Se han creado constructores. 
		&
		%PONDERACIÓN
		 \\
		 \cline{2-3}
		& \label{RA4:CEe}e) Se han desarrollado programas que instancien y utilicen objetos de las clases creadas anteriormente. 
		&
		%PONDERACIÓN
		 \\
		 \cline{2-3}
		& \label{RA4:CEf}f) Se han utilizado mecanismos para controlar la visibilidad de las clases y de sus miembros. 
		&
		%PONDERACIÓN
		 \\
		 \cline{2-3}
		& \label{RA4:CEg}g) Se han definido y utilizado clases heredadas. 
		&
		%PONDERACIÓN
		 \\
		 \cline{2-3}
		& \label{RA4:CEh}h) Se han creado y utilizado métodos estáticos. 
		&
		%PONDERACIÓN
		 \\
		 \cline{2-3}
		& \label{RA4:CEi}i) Se han creado y utilizado conjuntos y librerías de clases. 
		&
		%PONDERACIÓN
		 \\
		 \cline{2-3}
		\hline
		% Siguiente fila
\pagebreak\multirow{8}{*}{\parbox{3cm}{\vspace{0.4cm}\textbf{RA5 (\pondRAp{5})}\label{RA5}:\\ Realiza operaciones de entrada y salida de información, utilizando procedimientos específicos del lenguaje y librerías de clases.}}
		& \label{RA5:CEa}a) Se ha utilizado la consola para realizar operaciones de entrada y salida de información. 
		&
		%PONDERACIÓN
		 \\
		 \cline{2-3}
		& \label{RA5:CEb}b) Se han aplicado formatos en la visualización de la información. 
		&
		%PONDERACIÓN
		 \\
		 \cline{2-3}
		& \label{RA5:CEc}c) Se han reconocido las posibilidades de entrada / salida del lenguaje y las librerías asociadas. 
		&
		%PONDERACIÓN
		 \\
		 \cline{2-3}
		& \label{RA5:CEd}d) Se han utilizado ficheros para almacenar y recuperar información. 
		&
		%PONDERACIÓN
		 \\
		 \cline{2-3}
		& \label{RA5:CEe}e) Se han creado programas que utilicen diversos métodos de acceso al contenido de los ficheros. 
		&
		%PONDERACIÓN
		 \\
		 \cline{2-3}
		& \label{RA5:CEf}f) Se han utilizado las herramientas del entorno de desarrollo para crear interfaces gráficos de usuario simples. 
		&
		%PONDERACIÓN
		 \\
		 \cline{2-3}
		& \label{RA5:CEg}g) Se han programado controladores de eventos. 
		&
		%PONDERACIÓN
		 \\
		 \cline{2-3}
		& \label{RA5:CEh}h) Se han escrito programas que utilicen interfaces gráficos para la entrada y salida de información. 
		&
		%PONDERACIÓN
		 \\
		 \cline{2-3}
		\hline
		% Siguiente fila
\multirow{10}{*}{\parbox{3cm}{\vspace{0.4cm}\textbf{RA6 (\pondRAp{6})}\label{RA6}:\\ Escribe programas que manipulen información seleccionando y utilizando tipos avanzados de datos.}}
		& \label{RA6:CEa}a) Se han escrito programas que utilicen matrices (arrays). 
		&
		%PONDERACIÓN
		 \\
		 \cline{2-3}
		& \label{RA6:CEb}b) Se han reconocido las librerías de clases relacionadas con tipos de datos avanzados. 
		&
		%PONDERACIÓN
		 \\
		 \cline{2-3}
		& \label{RA6:CEc}c) Se han utilizado listas para almacenar y procesar información. 
		&
		%PONDERACIÓN
		 \\
		 \cline{2-3}
		& \label{RA6:CEd}d) Se han utilizado iteradores para recorrer los elementos de las listas. 
		&
		%PONDERACIÓN
		 \\
		 \cline{2-3}
		& \label{RA6:CEe}e) Se han reconocido las características y ventajas de cada una de las colecciones de datos disponibles. 
		&
		%PONDERACIÓN
		 \\
		 \cline{2-3}
		& \label{RA6:CEf}f) Se han creado clases y métodos genéricos. 
		&
		%PONDERACIÓN
		 \\
		 \cline{2-3}
		& \label{RA6:CEg}g) Se han utilizado expresiones regulares en la búsqueda de patrones en cadenas de texto. 
		&
		%PONDERACIÓN
		 \\
		 \cline{2-3}
		& \label{RA6:CEh}h) Se han identificado las clases relacionadas con el tratamiento de documentos escritos en diferentes lenguajes de intercambio de datos. 
		&
		%PONDERACIÓN
		 \\
		 \cline{2-3}
		& \label{RA6:CEi}i) Se han realizado programas que realicen manipulaciones sobre documentos escritos en diferentes lenguajes de intercambio de datos. 
		&
		%PONDERACIÓN
		 \\
		 \cline{2-3}
		& \label{RA6:CEj}j) Se han utilizado operaciones agregadas para el manejo de información almacenada en colecciones. 
		&
		%PONDERACIÓN
		 \\
		 \cline{2-3}
		\hline
		% Siguiente fila
\multirow{10}{*}{\parbox{3cm}{\vspace{0.4cm}\textbf{RA7 (\pondRAp{7})}\label{RA7}:\\ Desarrolla programas aplicando características avanzadas de los lenguajes orientados a objetos y del entorno de programación.}}
		& \label{RA7:CEa}a) Se han identificado los conceptos de herencia, superclase y subclase. 
		&
		%PONDERACIÓN
		 \\
		 \cline{2-3}
		& \label{RA7:CEb}b) Se han utilizado modificadores para bloquear y forzar la herencia de clases y métodos. 
		&
		%PONDERACIÓN
		 \\
		 \cline{2-3}
		& \label{RA7:CEc}c) Se ha reconocido la incidencia de los constructores en la herencia. 
		&
		%PONDERACIÓN
		 \\
		 \cline{2-3}
		& \label{RA7:CEd}d) Se han creado clases heredadas que sobrescriben la implementación de métodos de la superclase. 
		&
		%PONDERACIÓN
		 \\
		 \cline{2-3}
		& \label{RA7:CEe}e) Se han diseñado y aplicado jerarquías de clases. 
		&
		%PONDERACIÓN
		 \\
		 \cline{2-3}
		& \label{RA7:CEf}f) Se han probado y depurado las jerarquías de clases. 
		&
		%PONDERACIÓN
		 \\
		 \cline{2-3}
		& \label{RA7:CEg}g) Se han realizado programas que implementen y utilicen jerarquías de clases. 
		&
		%PONDERACIÓN
		 \\
		 \cline{2-3}
		& \label{RA7:CEh}h) Se ha comentado y documentado el código. 
		&
		%PONDERACIÓN
		 \\
		 \cline{2-3}
		& \label{RA7:CEi}i) Se han identificado y evaluado los escenarios de uso de interfaces. 
		&
		%PONDERACIÓN
		 \\
		 \cline{2-3}
		& \label{RA7:CEj}j) Se han identificado y evaluado los escenarios de utilización de la herencia y la composición. 
		&
		%PONDERACIÓN
		 \\
		 \cline{2-3}
		\hline
		% Siguiente fila
\pagebreak\multirow{8}{*}{\parbox{3cm}{\vspace{0.4cm}\textbf{RA8 (\pondRAp{8})}\label{RA8}:\\ Utiliza bases de datos orientadas a objetos, analizando sus características y aplicando técnicas para mantener la persistencia de la información.}}
		& \label{RA8:CEa}a) Se han identificado las características de las bases de datos orientadas a objetos. 
		&
		%PONDERACIÓN
		 \\
		 \cline{2-3}
		& \label{RA8:CEb}b) Se ha analizado su aplicación en el desarrollo de aplicaciones mediante lenguajes orientados a objetos. 
		&
		%PONDERACIÓN
		 \\
		 \cline{2-3}
		& \label{RA8:CEc}c) Se han instalado sistemas gestores de bases de datos orientados a objetos. 
		&
		%PONDERACIÓN
		 \\
		 \cline{2-3}
		& \label{RA8:CEd}d) Se han clasificado y analizado los distintos métodos soportados por los sistemas gestores para la gestión de la información almacenada. 
		&
		%PONDERACIÓN
		 \\
		 \cline{2-3}
		& \label{RA8:CEe}e) Se han creado bases de datos y las estructuras necesarias para el almacenamiento de objetos. 
		&
		%PONDERACIÓN
		 \\
		 \cline{2-3}
		& \label{RA8:CEf}f) Se han programado aplicaciones que almacenen objetos en las bases de datos creadas. 
		&
		%PONDERACIÓN
		 \\
		 \cline{2-3}
		& \label{RA8:CEg}g) Se han realizado programas para recuperar, actualizar y eliminar objetos de las bases de datos. 
		&
		%PONDERACIÓN
		 \\
		 \cline{2-3}
		& \label{RA8:CEh}h) Se han realizado programas para almacenar y gestionar tipos de datos estructurados, compuestos y relacionados. 
		&
		%PONDERACIÓN
		 \\
		 \cline{2-3}
		\hline
		% Siguiente fila
\multirow{6}{*}{\parbox{3cm}{\vspace{0.4cm}\textbf{RA9 (\pondRAp{9})}\label{RA9}:\\ Gestiona información almacenada en bases de datos manteniendo la integridad y consistencia de los datos.}}
		& \label{RA9:CEa}a) Se han identificado las características y métodos de acceso a sistemas gestores de bases de datos. 
		&
		%PONDERACIÓN
		 \\
		 \cline{2-3}
		& \label{RA9:CEb}b) Se han programado conexiones con bases de datos. 
		&
		%PONDERACIÓN
		 \\
		 \cline{2-3}
		& \label{RA9:CEc}c) Se ha escrito un código para almacenar información en bases de datos. 
		&
		%PONDERACIÓN
		 \\
		 \cline{2-3}
		& \label{RA9:CEd}d) Se han creado programas para recuperar y mostrar información almacenada en bases de datos. 
		&
		%PONDERACIÓN
		 \\
		 \cline{2-3}
		& \label{RA9:CEe}e) Se han efectuado borrados y modificaciones sobre la información almacenada. 
		&
		%PONDERACIÓN
		 \\
		 \cline{2-3}
		& \label{RA9:CEf}f) Se han creado aplicaciones que muestren la información almacenada en bases de datos. 
		&
		%PONDERACIÓN
		 \\
		 \cline{2-3}
		& \label{RA9:CEg}g) Se han creado aplicaciones para gestionar la información presente en bases de datos. 
		&
		%PONDERACIÓN
		 \\
		 \cline{2-3}
\end{longtable}
\newpage