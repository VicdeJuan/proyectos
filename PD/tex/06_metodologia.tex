\section{Principios metodológicos}

La metodología propuesta en esta programación se fundamenta en el enfoque constructivista, que toma como punto de partida los conocimientos previos del alumnado. Este enfoque requiere realizar una evaluación inicial del nivel de desarrollo y competencias existentes mediante actividades diagnósticas o pruebas de nivel. De este modo, las experiencias educativas se vuelven significativas, al conectar el aprendizaje nuevo con el conocimiento que los estudiantes ya poseen.

El aprendizaje de destrezas profesionales demanda una participación activa del alumnado, por lo que se proporcionarán actividades variadas y eminentemente prácticas que permitan a los estudiantes entrenar y perfeccionar habilidades profesionales en contextos reales o simulados. Se alternarán tareas individuales y actividades cooperativas, fomentando la autonomía y al mismo tiempo el aprendizaje colaborativo.

Se aplicará el \index{Diseño Universal para el Aprendizaje (DUA)} Diseño Universal para el Aprendizaje (DUA), mediante una diversidad de actividades y tareas que favorecen la participación plena de todo el alumnado, atendiendo así a las diferentes maneras de acceder al conocimiento y estilos de aprendizaje.

La metodología será interactiva y contextualizada, adecuando siempre el lenguaje y las actividades a las características particulares del alumnado. Las explicaciones teóricas se abordarán mediante una metodología inductiva, la cual parte de la presentación inicial de retos o problemas prácticos. Esta metodología estimula la curiosidad natural del alumnado, favorece la implicación activa en la búsqueda de soluciones y facilita la comprensión profunda de los conceptos. Tras este proceso inductivo inicial, los contenidos se formalizan con explicaciones teóricas que consolidan y sistematizan lo aprendido.

En cuanto a las entregas, predominan las individuales para fomentar la responsabilidad personal y la autoevaluación del alumno. Sin embargo, ocasionalmente se integrarán entregas grupales con problemas específicos que exijan una auténtica cooperación y trabajo conjunto. Incluso en actividades predominantemente individuales, se promoverán espacios de interacción grupal para que los estudiantes puedan intercambiar ideas, enriquecer su aprendizaje y practicar competencias sociales.

Las actividades estarán estructuradas según su función en el proceso de aprendizaje:

\begin{itemize}
    \item \textbf{Actividades de iniciación y motivación:} Estimulan el interés inicial del alumnado.
    \item \textbf{Actividades de desarrollo y aprendizaje:} Introducen y profundizan en nuevos contenidos.
    \item \textbf{Actividades procedimentales:} Desarrollan competencias técnicas específicas.
    \item \textbf{Actividades de consolidación:} Refuerzan y afianzan el aprendizaje.
    \item \textbf{Actividades de refuerzo:} Dan apoyo adicional a estudiantes que requieren mayor atención.
\end{itemize}



\section{Recursos}



Finalmente, se integrarán herramientas TIC como parte esencial de la metodología educativa:

\begin{itemize}
    \item Plataforma educativa (EducaMadrid) para gestión de contenidos y tareas, foros para la comunicación, cuestionarios interactivos, insignias digitales como reconocimiento del aprendizaje competencial, tareas colaborativas mediante wikis y glosarios interactivos para construir conocimiento compartido.
    \item Herramientas interactivas como Kahoot para gamificar contenidos teóricos.
    \item Jueces online y repositorios de código (Git) para el desarrollo práctico y colaborativo.
    \item Uso del ordenador del profesor en red local para entregas puntuales que deban realizarse sin conexión a internet.
\end{itemize}





\section{Medidas de atención a la diversidad para alumnos con necesidad específica de apoyo educativo}

Las medidas de atención a la diversidad para alumnado ACNEAE estarán enmarcadas dentro de lo previsto en el artículo 41 de la Orden 893/2022, de 21 de abril, de la Consejería de Educación, Universidades, Ciencia y Portavocía.

Desde la Programación se responde a la diversidad de capacidades, intereses, motivaciones y estilos particulares de aprender adoptando las siguientes medidas (que en ningún caso supondrán adaptaciones curriculares significativas):

