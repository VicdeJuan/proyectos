% Contenido general del marco normativo vigente
 
 \section*{Marco normativo vigente}
 \label{sec:marco-normativo}
 
 \subsection*{Estatal}
 \label{sec:normativa-estatal}

 \begin{itemize}
     \item \label{ley:lomloe_l}\namedref{ley:lomloe}{Ley Orgánica 3/2020}, de 29 de diciembre, por la que se modifica la \label{ley:loe_l}\namedref{ley:loe}{Ley Orgánica 2/2006} de Educación.  Establece el marco educativo actual, orientado a un enfoque competencial y por resultados de aprendizaje.
 
     \item \label{ley:nueva-fp_l}\namedref{ley:nueva-fp}{Ley Orgánica 3/2022} Ley Orgánica 3/2022, de 31 de marzo, de Ordenación e Integración de la Formación Profesional.
 
    \item \label{rd:278-2023_l}\namedref{rd:659-2023}{Real Decreto 278/2023}, de 11 de abril, por el que se establece el calendario de implantación del Sistema de Formación Profesional establecido por la Ley Orgánica 3/2022, de 31 de marzo, de ordenación e integración de la Formación Profesional.

     \item \label{rd:659-2023_l}\namedref{rd:659-2023}{Real Decreto 659/2023}, de 18 de julio, por el que se desarrolla la ordenación del Sistema de Formación Profesional.
 
     \item \label{rd:500-2024_l}\namedref{rd:500-2024}{Real Decreto 500/2024} Norma reciente que actualiza el marco curricular y organizativo de la Formación Profesional a nivel estatal, en línea con los principios de la LOMLOE y la nueva ley de FP.
 
     \item  \label{rd:450-2010_l}\namedref{rd:450-2010}{Real Decreto 450/2010} de 16 de abril, establece el título de Técnico Superior en Desarrollo de Aplicaciones Multiplataforma 
     y
     \label{rd:405-2023_l}\namedref{rd:405-2023}{Real Decreto 405/2023}, de 29 de mayo, actualiza dicho título y fija las enseñanzas mínimas del ciclo.
     \begin{itemize}
         \item El  \lawref{rd:405-2023}, en su anexo, establece el currículo básico del módulo 0485 Programación, con los resultados de aprendizaje, criterios de evaluación y contenidos mínimos que se deben cumplir a nivel estatal. Esta programación se ajusta íntegramente a dichos mínimos curriculares.
         \item El \lawref{rd:450-2010} mantiene el perfil profesional, competencia general, entorno profesional, listado de módulos profesionales, espacio y equipamientos, profesorado, accesos y vinculación a otros estudios, y correspondencia de módulos profesionales con las unidades de competencia.
     \end{itemize}
 \end{itemize}
 
 \subsection*{Autonómica}
 \label{sec:normativa-autonomica}
 Para todas las enseñanzas de formación profesional:
 \begin{itemize}
     \item \label{dec:63-2019_l}\namedref{dec:63-2019}{Decreto 63/2019} Decreto 63/2019, de 16 de julio, por el que se regula la ordenación y organización de la formación profesional de la Comunidad de Madrid (pendiente de publicación de nuevo decreto que lo derogará).
 
     \item \namedref{ord:893-2022}{Orden 893/2022} \label{ord:3413-2022_l} (modificada por \label{ord:3413-2022_l}\namedref{ord:3413-2022}{Orden 3413/2022}, de 21 de abril, por la que se regulan los procedimientos relacionados con la organización, la matrícula, la evaluación y acreditación académica de las enseñanzas de formación profesional del sistema educativo en la Comunidad de Madrid.  

     Modificada por Orden de 15 de noviembre de 2022 (pendiente de publicación de nueva orden que la derogará).
 
     \item \label{res:4jun2024_l}\namedref{res:4jun2024}{Resolución de 4 de junio de 2024}, de la Dirección General de Educación Secundaria, Formación Profesional y Régimen Especial, por la que se dictan instrucciones sobre la ordenación y la organización de los grados D y E de formación profesional en el curso académico 2024-2025.
 
     \item \label{dec:103-2024_l}\namedref{dec:103-2024}{Decreto 103/2024}, de 13 de noviembre, del Consejo de Gobierno, por el que se modifican setenta y seis decretos por los que se establecen para la Comunidad de Madrid planes de estudios de ciclos formativos de grado superior.
 
     \item \label{dec:3-2011_l}\namedref{dec:3-2011}{Decreto 3/2011}, de 13 de enero (BOCM 31/01/2011), que establece el currículo del ciclo DAM en Madrid, basado en el anterior \lawref{rd:450-2010}. Aunque está en proceso de actualización según la nueva ley de FP, sigue vigente como referencia para la organización del ciclo (distribución horaria, módulos profesionales, etc.). Este decreto enfatiza que el currículo oficial debe concretarse en programaciones didácticas adaptadas a cada centro, permitiendo adecuaciones particulares en función de los recursos, sin suprimir objetivos ni la competencia general del título.
 \end{itemize}

