
\section{Resumen de conceptos}

Conceptos, de más general e importante, a más concreto.
\begin{itemize}

\item \index{Competencia general del título}
\textbf{Competencia general del título}: Describe las funciones profesionales del perfil profesional del título. 

\item \index{Competencias personales y para la empleabilidad}
\textbf{Competencias personales y para la empleabilidad}: Describen el conjunto de conocimientos y destrezas que permiten responder a los requerimientos del sector productivo del título. 

\item \index{Objetivos}
\textbf{Objetivos}: Son los logros que el estudiante alcanzará al finalizar el ciclo, como resultado de las experiencias de enseñanza-aprendizaje intencionalmente planificadas. 

\item \index{Estándares de competencia}
\textbf{Estándares de competencia}: constituyen la unidad básica para el diseño de la formación y para la acreditación de competencias profesionales adquiridas por experiencia laboral u otras vías no formales o informales.

\item \index{Resultados de aprendizaje}
\textbf{Resultados de aprendizaje}: Son lo que se espera que un estudiante sea capaz de hacer una vez finalizado el proceso de aprendizaje. Son como los objetivos a alcanzar en cada módulo del ciclo. Son el eje vertebrador de la programación.

\item \index{Criterios de evaluación}
\textbf{Criterios de evaluación}: Describen los aspectos, en extensión, de los resultados de aprendizaje. Permiten conocer si se han alcanzado y con qué nivel de logro los resultados de aprendizaje. 

\item \index{Saberes mínimos}
\textbf{Saberes mínimos}: Son los conocimientos (conceptos, procedimientos, habilidades, destrezas y actitudes) asociados al desempeño profesional que, mediante las actividades de enseñanza-aprendizaje, permiten alcanzar los resultados de aprendizaje. 

\item \index{Metodología didáctica}
\textbf{Metodología didáctica}: el conjunto de estrategias, procedimientos y acciones organizadas y planificadas por el profesorado, con la finalidad de posibilitar el aprendizaje del alumnado y el logro de los resultados de aprendizaje.
\end{itemize}
