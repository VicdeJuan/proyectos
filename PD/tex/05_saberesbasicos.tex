\begin{itemize}[itemsep=0.1em, topsep=0.1em]
\item Introducción a la programación:
\subitem Datos, algoritmos y programas.
\subitem Paradigmas de programación.
\subitem Lenguajes de programación.
\subitem Herramientas y entornos para el desarrollo de programas.
\subitem Errores y calidad de los programas.
\item Introducción a la orientación a objetos:
\subitem Clases. Atributos, métodos y visibilidad
\subitem Objetos. Estado, comportamiento e identidad. Mensajes.
\subitem Encapsulado. Visibilidad.
\subitem Relaciones entre clases.
\subitem Principios básicos de la orientación a objetos.
\item Identificación de los elementos de un programa informático:
\subitem Estructura y bloques fundamentales.
\subitem Identificadores.
\subitem Palabras reservadas.
\subitem Variables. Declaración, inicialización y utilización. Almacenamiento en memoria.
\subitem Tipos de datos.
\subitem Literales.
\subitem Constantes.
\subitem Operadores y expresiones. Precedencia de operadores
\subitem Conversiones de tipo. Implícitas y explicitas (casting).
\subitem Comentarios.
\item Utilización de objetos:
\subitem Características de los objetos.
\subitem Constructores.
\subitem Instanciación de objetos. Declaración y creación.
\subitem Utilización de métodos. Parámetros y valores de retorno.
\subitem Utilización de propiedades.
\subitem Utilización de métodos estáticos.
\subitem Almacenamiento en memoria. Tipos básicos vs objetos.
\subitem Destrucción de objetos y liberación de memoria.
\item Uso de estructuras de control:
\subitem Estructuras de selección.
\subitem Estructuras de repetición.
\subitem Estructuras de salto.
\item Desarrollo de clases:
\subitem Concepto de clase.
\subitem Estructura y miembros de una clase.
\subitem Creación de atributos. Declaración e inicialización.
\subitem Creación de métodos. Declaración, argumentos y valores de retorno.
\subitem Creación de constructores.
\subitem Ámbito de atributos y variables.
\subitem Sobrecarga de métodos.
\subitem Visibilidad. Modificadores de clase, de atributos y de métodos.
\subitem Paso de parámetros. Paso por valor y paso por referencia.
\subitem Utilización de clases y objetos.
\subitem Utilización de clases heredadas.
\subitem Librerías y paquetes de clases. Utilización y creación.
\subitem Documentación sobre librerías y paquetes de clases.
\item Aplicación de las estructuras de almacenamiento:
\subitem Estructuras.
\subitem Arrays unidimensionales y multidimensionales:  Declaración.  Creación de arrays unidimensionales y multidimensionales.  Inicialización  Acceso a elementos.  Recorridos, búsquedas y ordenaciones.
\subitem Cadenas de caracteres: Declaración.  Creación de cadenas de caracteres.  Inicialización Operaciones. Acceso a elementos, conversiones, concatenación.
\item Utilización avanzada de clases:
\subitem Relaciones entre clases. Composición de clases.
\subitem Herencia. Concepto y tipos (simple y múltiple).
\subitem Superclases y subclases.
\subitem Constructores y herencia.
\subitem Modificadores en clases, atributos y métodos.
\subitem Sobreescritura de métodos.
\subitem Clases y métodos abstractos y finales.
\subitem Interfaces. Clases abstractas vs. Interfaces.
\subitem Polimorfismo: Concepto.  Polimorfismo en tiempo de compilación (sobrecarga) y polimorfismo en tiempo de ejecución (ligadura dinámica).  Comprobación estática y dinámica de tipos.
\subitem Conversiones de tipos entre objetos (casting).
\subitem Clases y tipos genéricos o parametrizados.
\item Control y manejo de excepciones:
\subitem Excepciones. Concepto.
\subitem Jerarquías de excepciones.
\subitem Manejo de excepciones:  Captura de excepciones.  Propagar excepciones.  Lanzar excepciones.  Crear clases de excepciones.
\item Colecciones de datos:
\subitem Tipos de colecciones (listas, pilas, colas, tablas).
\subitem Jerarquías de colecciones.
\subitem Operaciones con colecciones. Acceso a elementos y recorridos.
\subitem Uso de clases y métodos genéricos.
\item Lectura y escritura de información:
\subitem Flujos (streams): Tipos de flujos. Flujos de bytes y de caracteres. Clases relativas a flujos. Jerarquías de clases. Utilización de flujos.
\subitem Entrada/salida estándar: Entrada desde teclado. Salida a pantalla.
\subitem Almacenamiento de información en ficheros: Ficheros de datos. Registros. Apertura y cierre de ficheros. Modos de acceso. Escritura y lectura de información en ficheros. Almacenamiento de objetos en ficheros.Persistencia. Serialización. Utilización de los sistemas de ficheros. Creación y eliminación de ficheros y directorios.
\subitem Interfaces gráficos de usuario simples. Concepto de evento. Creación de controladores de eventos.
\item Gestión de bases de datos relacionales:
\subitem Interfaces de programación de acceso a bases de datos.
\subitem Establecimiento de conexiones.
\subitem Recuperación de información.
\subitem Manipulación de la información.
\subitem Ejecución de consultas sobre la base de datos.
\item Mantenimiento de la persistencia de los objetos:
\subitem Bases de datos orientadas a objetos.
\subitem Características de las bases de datos orientadas a objetos.
\subitem Instalación del gestor de bases de datos.
\subitem Creación de bases de datos.
\subitem Mecanismos de consulta.
\subitem El lenguaje de consultas: sintaxis, expresiones, operadores.
\subitem Recuperación, modificación y borrado de información.
\subitem Tipos de datos objeto; atributos y métodos.
\subitem Tipos de datos colección.
\end{itemize}