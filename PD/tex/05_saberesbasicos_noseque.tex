\section{Saberes básicos}

A continuación, se presentan los \index{Saberes Básicos} saberes básicos definidos en \lawref{dec:3-2011} (concreción autonómica del \lawref{rd:450-2010} y de \lawref{rd:405-2023}).

Estos saberes básicos serán el medio con el que conseguir los aprendizajes concretos definidos por los Resultados de Aprendizaje que, a su vez, son los que garantizan la consecución de las competencias de módulo permitiendo, de esta manera, conseguir adquirir todas las competencias de ciclo.

Estos son los contenidos que se tratarán en cada una de las Unidades Didácticas.

\begin{itemize}[itemsep=0.1em, topsep=0.1em]

\item\emph{Introducción a la orientación a objetos:}
Clases. Atributos, métodos y visibilidad
Objetos. Estado, comportamiento e identidad. Mensajes.
Encapsulado. Visibilidad.
Relaciones entre clases.
Principios básicos de la orientación a objetos.

\item\emph{Utilización de objetos:}
Características de los objetos.
Constructores.
Instanciación de objetos. Declaración y creación.
Utilización de métodos. Parámetros y valores de retorno.
Utilización de propiedades.
Utilización de métodos estáticos.
Almacenamiento en memoria. Tipos básicos vs objetos.
Destrucción de objetos y liberación de memoria.


\item\emph{Desarrollo de clases:}
Concepto de clase.
Estructura y miembros de una clase.
Creación de atributos. Declaración e inicialización.
Creación de métodos. Declaración, argumentos y valores de retorno.
Creación de constructores.
Ámbito de atributos y variables.
Sobrecarga de métodos.
Visibilidad. Modificadores de clase, de atributos y de métodos.
Paso de parámetros. Paso por valor y paso por referencia.
Utilización de clases y objetos.
Utilización de clases heredadas.
Librerías y paquetes de clases. Utilización y creación.
Documentación sobre librerías y paquetes de clases.

\item\emph{Aplicación de las estructuras de almacenamiento:}






\end{itemize}