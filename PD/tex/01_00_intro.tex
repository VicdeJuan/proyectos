
La presente programación didáctica corresponde al módulo profesional de Programación, perteneciente al primer curso del Ciclo Formativo de Grado Superior (CFGS) en Desarrollo de Aplicaciones Multiplataforma (DAM).

El módulo de Programación es fundamental dentro del ciclo formativo, ya que establece las bases conceptuales y procedimentales necesarias para desarrollar aplicaciones multiplataforma. 

\label{sec:intro_00_conocimiento_previo}
En cuanto a su planteamiento didáctico, se parte de la premisa de que todo el alumnado comienza desde cero, independientemente de los conocimientos previos que pudiera tener. Es posible que algunos estudiantes lleguen al ciclo con cierta experiencia en programación adquirida de forma autodidacta o en estudios anteriores, pero en muchos casos estos aprendizajes responden a enfoques parciales, carentes de rigor o orientados únicamente a que “funcione como sea”. En este módulo se persigue un conocimiento sólido, bien estructurado y con fundamentos teóricos y prácticos robustos. Nuestro objetivo es que el alumnado adquiera no solo la capacidad de programar, sino de hacerlo con criterio profesional, buenas prácticas y capacidad de análisis y resolución de problemas, sentando así las bases para el resto de módulos del ciclo.

La metodología empleada busca atender la diversidad del grupo y alinearse con las oportunidades y limitaciones del contexto socioeconómico local, destacando un enfoque activo y colaborativo del aprendizaje, con énfasis en el desarrollo de proyectos reales y la resolución práctica de problemas mediante el uso de herramientas tecnológicas profesionales.

Esta programación didáctica se elabora como un documento técnico-pedagógico que establece los objetivos, contenidos, metodología, criterios e instrumentos de evaluación, y medidas de atención a la diversidad que se aplicarán a lo largo del curso. Su finalidad es servir como guía estructurada para garantizar una enseñanza coherente, rigurosa y adaptada a las necesidades reales del alumnado y del entorno. Asimismo, este documento pretende ser un referente tanto para el profesorado como para el propio equipo docente del ciclo, favoreciendo la coordinación, la reflexión compartida y la mejora continua del proceso de enseñanza-aprendizaje.

Para garantizar la coherencia con el marco institucional y normativo del centro, esta programación didáctica se elabora teniendo en cuenta diversos documentos clave que rigen la organización, funcionamiento y orientación educativa del instituto. Entre ellos se incluyen:

\begin{itemize}
\item El Proyecto Educativo del Centro (PEC), que define los principios, valores y objetivos educativos generales.
\item La Programación General Anual (PGA), donde se concretan los objetivos operativos y las actuaciones planificadas para el curso académico.
\item El Plan de Acción Tutorial (PAT), especialmente relevante para la coordinación con los tutores en lo relativo al seguimiento académico y personal del alumnado.
\item El Plan de Atención a la Diversidad (PAD), que establece las directrices para la inclusión y atención específica al alumnado con necesidades educativas especiales o específicas.
\item El Plan de Convivencia, que orienta la gestión de la convivencia escolar y el desarrollo de un clima positivo en el aula.
\item El Plan Digital de Centro, que proporciona el marco para la integración de las tecnologías digitales en la práctica docente.
\item Los currículos oficiales de los ciclos formativos, aprobados por la Comunidad de Madrid, que definen los resultados de aprendizaje, criterios de evaluación y contenidos del módulo.
Todos estos documentos sirven de referencia para asegurar que la práctica docente en este módulo se alinea con las directrices generales del centro y con la normativa educativa vigente.
\end{itemize}

A continuación, destacamos el marco normativo específico, tanto autonómico como estatal que vertebra esta programación didáctica.