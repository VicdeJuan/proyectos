


Referencia al Real Decreto de enseñanzas mínimas: El módulo Programación tiene una duración mínima de 135 horas según el \lawref{rd:405-2023}, lo que en el currículo de Madrid (\lawref{dec:3-2011}) equivale a 270 horas distribuidas en el primer curso (aprox. 8 horas semanales). Esta distribución horaria y curricular es la base para la temporalización y secuenciación de unidades didácticas que se detalla más adelante.

\todo{Mención a las FCEs.}


% Contenido de los objetivos

\section{Competencias personales y para la empleabilidad a adquirir con el módulo}
\subsection{Competencias del ciclo}

Las competencias profesionales, personales y sociales de este título son las que se relacionan a continuación:
\begin{enumerate}[label=\alph*)]
%a) 
\item Configurar y explotar sistemas informáticos, adaptando la configuración lógica del sistema según las necesidades de uso y los criterios establecidos.
%b) 
\item Aplicar técnicas y procedimientos relacionados con la seguridad en sistemas, servicios y aplicaciones, cumpliendo el plan de seguridad.
%c) 
\item Gestionar bases de datos, interpretando su diseño lógico y verificando integridad, consistencia, seguridad y accesibilidad de los datos.
%d) 
\item Gestionar entornos de desarrollo adaptando su configuración en cada caso para permitir el desarrollo y despliegue de aplicaciones.
%e) 
\item \label{comp:01e} Desarrollar aplicaciones multiplataforma con acceso a bases de datos utilizando lenguajes, librerías y herramientas adecuados a las especificaciones.
%f) 
\item Desarrollar aplicaciones implementando un sistema completo de formularios e informes que permitan gestionar de forma integral la información almacenada.
%g) 
\item Integrar contenidos gráficos y componentes multimedia en aplicaciones multiplataforma, empleando herramientas específicas y cumpliendo los requerimientos establecidos.
%h) 
\item Desarrollar interfaces gráficos de usuario interactivos y con la usabilidad adecuada, empleando componentes visuales estándar o implementando componentes visuales específicos.
%i) 
\item Participar en el desarrollo de juegos y aplicaciones en el ámbito del entretenimiento y la educación empleando técnicas, motores y entornos de desarrollo específicos.
%j) 
\item \label{comp:02j}Desarrollar aplicaciones para teléfonos móviles, tabletas y otros dispositivos inteligentes empleando técnicas y entornos de desarrollo específicos.
%k) 
\item Crear ayudas generales y sensibles al contexto, empleando herramientas específicas e integrándolas en sus correspondientes aplicaciones.
%l) 
\item Crear tutoriales, manuales de usuario, de instalación, de configuración y de administración, empleando herramientas específicas.
%m) 
\item Empaquetar aplicaciones para su distribución preparando paquetes auto instalables con asistentes incorporados.
%n) 
\item Desarrollar aplicaciones multiproceso y multihilo empleando librerías y técnicas de programación específicas.
%ñ) 
\item Desarrollar aplicaciones capaces de ofrecer servicios en red empleando mecanismos de comunicación.
%o) 
\item Participar en la implantación de sistemas ERP-CRM evaluando la utilidad de cada uno de sus módulos.
%p) 
\item Gestionar la información almacenada en sistemas ERP-CRM garantizando su integridad.
%q) 
\item Desarrollar componentes personalizados para un sistema ERP-CRM atendiendo a los requerimientos.
%r) 
\item Realizar planes de pruebas verificando el funcionamiento de los componentes software desarrollados, según las especificaciones.
%s) 
\item Desplegar y distribuir aplicaciones en distintos ámbitos de implantación verificando su comportamiento y realizando las modificaciones necesarias.
%t) 
\item \label{comp:03t} Establecer vidas eficaces de relación profesional y comunicación con sus superiores, compañeros y subordinados, respetando la autonomía y competencias de las distintas personas.
%u) 
\item Liderar situaciones colectivas que se puedan producir, mediando en conflictos personales y laborales, contribuyendo al establecimiento de un ambiente de trabajo agradable, actuando en todo momento de forma respetuosa y tolerante.
%v) 
\item Gestionar su carrera profesional, analizando las oportunidades de empleo, autoempleo y de aprendizaje.
%w) 
\item \label{comp:04w} Mantener el espíritu de innovación y actualización en el ámbito de su trabajo para adaptarse a los cambios tecnológicos y organizativos de su entorno profesional.
%x) 
\item Crear y gestionar una pequeña empresa, realizando un estudio de viabilidad de productos, de planificación de la producción y de comercialización.
%y) 
\item Participar de forma activa en la vida económica, social y cultural, con una actitud crítica y responsable.»
\end{enumerate}

\subsection{Competencias del módulo:}
% Contenido de competencias

A continuación listamos las competencias que, de acuerdo con el \lawref{rd:450-2010}, se deben obtener en la consecución del ciclo:

\begin{itemize}
\item[\ref{comp:01e}] Desarrollar aplicaciones multiplataforma con acceso a bases de datos utilizando lenguajes, librerías y herramientas adecuados a las especificaciones.
\item[\ref{comp:02j}] Desarrollar aplicaciones para teléfonos móviles, tabletas y otros dispositivos inteligentes empleando técnicas y entornos de desarrollo específicos.
\item[\ref{comp:03t}] Establecer vidas eficaces de relación profesional y comunicación con sus superiores, compañeros y subordinados, respetando la autonomía y competencias de las distintas personas.
\item[\ref{comp:04w}] Mantener el espíritu de innovación y actualización en el ámbito de su trabajo para adaptarse a los cambios tecnológicos y organizativos de su entorno profesional.
\end{itemize}
