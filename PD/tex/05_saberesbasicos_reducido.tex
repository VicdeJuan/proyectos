\section{Saberes básicos}

A continuación, se presentan los \index{Saberes Básicos} saberes básicos definidos en \lawref{dec:3-2011} (concreción autonómica del \lawref{rd:450-2010} y de \lawref{rd:405-2023}).

Estos saberes básicos serán el medio con el que conseguir los aprendizajes concretos definidos por los Resultados de Aprendizaje que, a su vez, son los que garantizan la consecución de las competencias de módulo permitiendo, de esta manera, conseguir adquirir todas las competencias de ciclo.

Estos son los contenidos que se tratarán en cada una de las Unidades Didácticas.

\begin{itemize}[itemsep=0.1em, topsep=0.1em]
\item\emph{Introducción a la programación:}
Datos, algoritmos y programas.
Paradigmas de programación.
Lenguajes de programación.
Herramientas y entornos para el desarrollo de programas.
Errores y calidad de los programas.
\item\emph{Introducción a la orientación a objetos:}
Clases. Atributos, métodos y visibilidad
Objetos. Estado, comportamiento e identidad. Mensajes.
Encapsulado. Visibilidad.
Relaciones entre clases.
Principios básicos de la orientación a objetos.
\item\emph{Identificación de los elementos de un programa informático:}
Estructura y bloques fundamentales.
Identificadores.
Palabras reservadas.
Variables. Declaración, inicialización y utilización. Almacenamiento en memoria.
Tipos de datos.
Literales.
Constantes.
Operadores y expresiones. Precedencia de operadores
Conversiones de tipo. Implícitas y explicitas (casting).
Comentarios.
\item\emph{Utilización de objetos:}
Características de los objetos.
Constructores.
Instanciación de objetos. Declaración y creación.
Utilización de métodos. Parámetros y valores de retorno.
Utilización de propiedades.
Utilización de métodos estáticos.
Almacenamiento en memoria. Tipos básicos vs objetos.
Destrucción de objetos y liberación de memoria.
\item\emph{Uso de estructuras de control:}
Estructuras de selección.
Estructuras de repetición.
Estructuras de salto.
\item\emph{Desarrollo de clases:}
Concepto de clase.
Estructura y miembros de una clase.
Creación de atributos. Declaración e inicialización.
Creación de métodos. Declaración, argumentos y valores de retorno.
Creación de constructores.
Ámbito de atributos y variables.
Sobrecarga de métodos.
Visibilidad. Modificadores de clase, de atributos y de métodos.
Paso de parámetros. Paso por valor y paso por referencia.
Utilización de clases y objetos.
Utilización de clases heredadas.
Librerías y paquetes de clases. Utilización y creación.
Documentación sobre librerías y paquetes de clases.
\item\emph{Aplicación de las estructuras de almacenamiento:}
Estructuras.
Arrays unidimensionales y multidimensionales:  Declaración.  Creación de arrays unidimensionales y multidimensionales.  Inicialización  Acceso a elementos.  Recorridos, búsquedas y ordenaciones.
Cadenas de caracteres: Declaración.  Creación de cadenas de caracteres.  Inicialización Operaciones. Acceso a elementos, conversiones, concatenación.
\item\emph{Utilización avanzada de clases:}
Relaciones entre clases. Composición de clases.
Herencia. Concepto y tipos (simple y múltiple).
Superclases y subclases.
Constructores y herencia.
Modificadores en clases, atributos y métodos.
Sobreescritura de métodos.
Clases y métodos abstractos y finales.
Interfaces. Clases abstractas vs. Interfaces.
Polimorfismo: Concepto.  Polimorfismo en tiempo de compilación (sobrecarga) y polimorfismo en tiempo de ejecución (ligadura dinámica).  Comprobación estática y dinámica de tipos.
Conversiones de tipos entre objetos (casting).
Clases y tipos genéricos o parametrizados.
\item\emph{Control y manejo de excepciones:}
Excepciones. Concepto.
Jerarquías de excepciones.
Manejo de excepciones:  Captura de excepciones.  Propagar excepciones.  Lanzar excepciones.  Crear clases de excepciones.
\item\emph{Colecciones de datos:}
Tipos de colecciones (listas, pilas, colas, tablas).
Jerarquías de colecciones.
Operaciones con colecciones. Acceso a elementos y recorridos.
Uso de clases y métodos genéricos.
\item\emph{Lectura y escritura de información:}
Flujos (streams): Tipos de flujos. Flujos de bytes y de caracteres. Clases relativas a flujos. Jerarquías de clases. Utilización de flujos.
Entrada/salida estándar: Entrada desde teclado. Salida a pantalla.
Almacenamiento de información en ficheros: Ficheros de datos. Registros. Apertura y cierre de ficheros. Modos de acceso. Escritura y lectura de información en ficheros. Almacenamiento de objetos en ficheros.Persistencia. Serialización. Utilización de los sistemas de ficheros. Creación y eliminación de ficheros y directorios.
Interfaces gráficos de usuario simples. Concepto de evento. Creación de controladores de eventos.
\item\emph{Gestión de bases de datos relacionales:}
Interfaces de programación de acceso a bases de datos.
Establecimiento de conexiones.
Recuperación de información.
Manipulación de la información.
Ejecución de consultas sobre la base de datos.
\item\emph{Mantenimiento de la persistencia de los objetos:}
Bases de datos orientadas a objetos.
Características de las bases de datos orientadas a objetos.
Instalación del gestor de bases de datos.
Creación de bases de datos.
Mecanismos de consulta.
El lenguaje de consultas: sintaxis, expresiones, operadores.
Recuperación, modificación y borrado de información.
Tipos de datos objeto; atributos y métodos.
Tipos de datos colección.
\end{itemize}