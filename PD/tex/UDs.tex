\section{Unidades Didácticas}
\subsection{UD1: Introducción a la Programación}

	\paragraph{RAs y CEs:}
	\begin{itemize}[itemsep=0.1em, topsep=0.1em]
		\item\RAUNOa
	\end{itemize}

	\paragraph{Saberes Básicos:} \emph{Introducción a la programación:}
		Datos, algoritmos y programas.
		Paradigmas de programación.
		Lenguajes de programación.
		Herramientas y entornos para el desarrollo de programas.
		Errores y calidad de los programas.



	\paragraph{Sesiones:} 
		\todo{completar}

	\paragraph{Instrumento de calificación:} 
		test moodle con aplicaciones prácticas de esta introducción

\newpage
\subsection{UD2: Básicos de programación estructurada}

	\paragraph{RAs y CEs} 
	\begin{itemize}[itemsep=0.1em, topsep=0.1em]
		\item\RAUNOb
		\item\RAUNOc
		\item\RAUNOd
		\item\RAUNOe
		\item\RAUNOf
		\item\RAUNOg
		\item\RAUNOh
		\item\RAUNOi
		\item\RACINCOa
		\item\RACINCOb
		\item\RACINCOc
	\end{itemize}


	\paragraph{Saberes básicos:} \emph{Identificación de los elementos de un programa informático:}
		Estructura y bloques fundamentales.
		Identificadores.
		Palabras reservadas.
		Variables. Declaración, inicialización y utilización. Almacenamiento en memoria.
		Tipos de datos.
		Literales.
		Constantes.
		Operadores y expresiones. Precedencia de operadores
		Conversiones de tipo. Implícitas y explicitas (casting).
		Comentarios.

		Entrada Salida


	\paragraph{Sesiones:} 
		\todo{completar}

	\paragraph{Instrumentos de calificación: } Ejercicios a resolver en clase, entrega de prácticas individuales, tipo test de moodle para identificar errores

\newpage
\subsection{UD3: Estructuras de control}

	\paragraph{RAs y CEs:}
	\begin{itemize}[itemsep=0.1em, topsep=0.1em]
		\item\RATRESa
		\item\RATRESb
		\item\RATRESc
		\item\RATRESe
		\item\RATRESf
		\item\RATRESg		
		\item\RACINCOa
		\item\RACINCOb
		\item\RACINCOc	
	\end{itemize}

	\paragraph{Contenidos:}
		Estructuras de selección.
		Estructuras de repetición.
		Estructuras de salto.


	\paragraph{Sesiones:}
		\todo{completar}

	\paragraph{Instrumentos de calificación:}
		Ejercicios de clase, entregas individuales de prácticas (una de selección, otra de repetición y otra combinada), tipo test para reconocer bucles y aprender a pensar [no probar si es un < o un <=], vídeo explicativo de cómo depurar con gdb



\newpage
\subsection{UD4: Programación modular}
	\paragraph{RAs y CEs:}
	\begin{itemize}[itemsep=0.1em, topsep=0.1em]
		\item\RAUNOf
		\item\RAUNOi			
		\item\RATRESe
		\item\RATRESf
		\item\RATRESg 
		\item\RACINCOa
		\item\RACINCOb
		\item\RACINCOc			
	\end{itemize}

	\paragraph{Contenidos:}
		Ninguno del currículo. Podríamos saltarla.

		Funciones, parámetros y valores de retorno.Paso por valor y referencia.


	\paragraph{Sesiones:}
		\todo{completar}
	
	\paragraph{Instrumentos de calificación:}


\newpage
\subsection{UD5: Estructuras de almacenamiento en C}
	\paragraph{RAs y CEs:}
	\begin{itemize}[itemsep=0.1em, topsep=0.1em]
		\item\RAUNOf
		\item\RAUNOi			
		\item\RATRESe
		\item\RATRESf
		\item\RATRESg
		\item\RASEISa
		\item\RACINCOa
		\item\RACINCOb
		\item\RACINCOc	
	\end{itemize}
 	
 		\paragraph{Contenidos:}
		Estructuras.
		Arrays unidimensionales y multidimensionales:  Declaración.  Creación de arrays unidimensionales y multidimensionales.  Inicialización  Acceso a elementos.  Recorridos, búsquedas y ordenaciones.
		Cadenas de caracteres: Declaración.  Creación de cadenas de caracteres.  Inicialización Operaciones. Acceso a elementos, conversiones, concatenación.

		\textbf{Opcional y extra: Gestión de la memoria dinámica en C}


	\paragraph{Sesiones:}
		\todo{completar}
	\paragraph{Instrumentos de calificación:}
		Registros de clase sobre la utilización del depurador (RA3f)
		Checklist de comentarios(RA3g)
		Prácticas individuales.
		Práctica grupal



\newpage
\subsection{UD6: Cambio de paradigma: La Orientación a Objetos}

	\paragraph{RAs y CEs:}
	\begin{itemize}[itemsep=0.1em, topsep=0.1em]
		\item\RAUNOf
		\item\RAUNOi		
		\item\RATRESe
		\item\RATRESf
		\item\RATRESg
		\item\RADOSa
		\item\RADOSb
		\item\RADOSc
		\item\RADOSd
		\item\RADOSe
		\item\RADOSf
		\item\RADOSh
		\item\RADOSi		
		\item\RACINCOa
		\item\RACINCOb
		\item\RACINCOc
	\end{itemize}

	\paragraph{Contenidos:}
		Clases. Atributos, métodos y visibilidad
		Objetos. Estado, comportamiento e identidad. Mensajes.
		Encapsulado. Visibilidad.
		Relaciones entre clases.
		Principios básicos de la orientación a objetos.

		\emph{Utilización de objetos:}
		Características de los objetos.
		Constructores.
		Instanciación de objetos. Declaración y creación.
		Utilización de métodos. Parámetros y valores de retorno.
		Utilización de propiedades.
		Utilización de métodos estáticos.
		Almacenamiento en memoria. Tipos básicos vs objetos.
		Destrucción de objetos y liberación de memoria.

		Entrada/salida estándar: Entrada desde teclado. Salida a pantalla.

	\paragraph{Sesiones:}
		\todo{completar}
	\paragraph{Instrumentos de calificación:}
		\todo{completar}



\newpage
\subsection{UD7: Colecciones de datos}

	\paragraph{RAs y CEs:}
	\begin{itemize}[itemsep=0.1em, topsep=0.1em]
		\item\RAUNOf
		\item\RAUNOi		
		\item\RATRESe
		\item\RATRESf
		\item\RATRESg
		\item\RACINCOa
		\item\RACINCOb
		\item\RACINCOc
		\item\RADOSc
		\item\RADOSg
		\item\RADOSh
		\item\RADOSi
		\item\RASEISa
		\item\RASEISb
		\item\RASEISc
		\item\RASEISd
		\item\RASEISe
		\item\RASEISh
		\item\RASEISi
		\item\RASEISj
		\item\RACINCOd
		\item\RACINCOe
	\end{itemize}
	
	\paragraph{Contenidos:}
		\emph{Colecciones de datos:}
		Tipos de colecciones (listas, pilas, colas, tablas).
		Jerarquías de colecciones.
		Operaciones con colecciones. Acceso a elementos y recorridos.

	\paragraph{Sesiones:}
		\todo{completar}
	\paragraph{Instrumentos de calificación:}
		Prácticas individuales. 
		Una práctica de JSON (RA6b,hi;RA5de)



\newpage
\subsection{UD8: POO Intermedia}

	\paragraph{RAs y CEs:}
	\begin{itemize}[itemsep=0.1em, topsep=0.1em]
		\item\RAUNOf
		\item\RAUNOi		
		\item\RATRESe
		\item\RATRESf
		\item\RATRESg
		\item\RACINCOa
		\item\RACINCOb
		\item\RACINCOc		
		\item RA2*
		\item\RACUATROa
		\item\RACUATROb
		\item\RACUATROc
		\item\RACUATROd
		\item\RACUATROe
		\item\RACUATROf
		\item\RACUATROg
		\item\RACUATROh
		\item\RACUATROi
		\item RA6(FALTA)
		\item RA7(FALTA)
	\end{itemize}



	\paragraph{Contenidos:}
		\emph{Utilización avanzada de clases:}
		Relaciones entre clases. Composición de clases.
		Herencia. Concepto y tipos (simple y múltiple).
		Superclases y subclases.
		Constructores y herencia.
		Modificadores en clases, atributos y métodos.
		Sobreescritura de métodos.

		Flujos (streams): Tipos de flujos. Flujos de bytes y de caracteres. Clases relativas a flujos. Jerarquías de clases. Utilización de flujos.
		Entrada/salida estándar: Entrada desde teclado. Salida a pantalla.


	\paragraph{Sesiones:}
		\todo{completar}
	\paragraph{Instrumentos de calificación:}
		\todo{completar}


\newpage
\subsection{UD9: POO Avanzada}

	\paragraph{RAs y CEs:}
	\begin{itemize}[itemsep=0.1em, topsep=0.1em]

		\item\RAUNOf
		\item\RAUNOi		
		\item\RATRESe
		\item\RATRESf
		\item\RATRESg
		\item\RACINCOa
		\item\RACINCOb
		\item\RACINCOc		
		\item RA2*
		\item\RACUATROa
		\item\RACUATROb
		\item\RACUATROc
		\item\RACUATROd
		\item\RACUATROe
		\item\RACUATROf
		\item\RACUATROg
		\item\RACUATROh
		\item\RACUATROi
		\item\RASEISf
		\item RA7*
	\end{itemize}


	\paragraph{Contenidos:}
		\emph{Utilización avanzada de clases:}
		Clases y métodos abstractos y finales.
		Interfaces. Clases abstractas vs. Interfaces.
		Polimorfismo: Concepto.  Polimorfismo en tiempo de compilación (sobrecarga) y polimorfismo en tiempo de ejecución (ligadura dinámica).  Comprobación estática y dinámica de tipos.
		Conversiones de tipos entre objetos (casting).
		Clases y tipos genéricos o parametrizados.

		Uso de clases y métodos genéricos. (del bloque Colecciones de datos)

		\todo{completar}
	\paragraph{Sesiones:}
		\todo{completar}
	\paragraph{Instrumentos de calificación:}
		\todo{completar}


\newpage
\subsection{UD10:Control y manejo de excepciones:}

	\paragraph{RAs y CEs:}
	\begin{itemize}[itemsep=0.1em, topsep=0.1em]
		\item\RATRESd
		\item\RATRESh
		\item\RATRESi		
	\end{itemize}
	\paragraph{Contenidos:}
		Excepciones. Concepto.
		Jerarquías de excepciones.
		Manejo de excepciones:  Captura de excepciones.  Propagar excepciones.  Lanzar excepciones.  Crear clases de excepciones.


	\paragraph{Sesiones:}
		\todo{completar}
	\paragraph{Instrumentos de calificación:}
		\todo{completar}




\newpage
\subsection{UD11: Lectura y escritura de información}

	\paragraph{RAs y CEs:}
	\begin{itemize}[itemsep=0.1em, topsep=0.1em]
		\item RA5*
	\end{itemize}

	\paragraph{Contenidos:}
		Almacenamiento de información en ficheros: Ficheros de datos. Registros. Apertura y cierre de ficheros. Modos de acceso. Escritura y lectura de información en ficheros. Almacenamiento de objetos en ficheros.Persistencia. Serialización. Utilización de los sistemas de ficheros. Creación y eliminación de ficheros y directorios.
		Interfaces gráficos de usuario simples. Concepto de evento. Creación de controladores de eventos.

	\paragraph{Sesiones:}
		\todo{completar}
	\paragraph{Instrumentos de calificación:}
		\todo{completar}


\newpage
\subsection{UD12: Gestión de bases de datos relacionales}

	\paragraph{RAs y CEs:}
	\begin{itemize}[itemsep=0.1em, topsep=0.1em]
		\item RA8*
		\item RA9*
	\end{itemize}

	\paragraph{Contenidos:}
		
		Interfaces de programación de acceso a bases de datos.
		Establecimiento de conexiones.
		Recuperación de información.
		Manipulación de la información.
		Ejecución de consultas sobre la base de datos.

	\paragraph{Sesiones:}
		\todo{completar}
	\paragraph{Instrumentos de calificación:}
		\todo{completar}



\newpage
\subsection{UD13: Mantenimiento de la persistencia de los objetos:}

	\paragraph{RAs y CEs:}
		\todo{completar}
	\paragraph{Contenidos:}
		Bases de datos orientadas a objetos.
		Características de las bases de datos orientadas a objetos.
		Instalación del gestor de bases de datos.
		Creación de bases de datos.
		Mecanismos de consulta.
		El lenguaje de consultas: sintaxis, expresiones, operadores.
		Recuperación, modificación y borrado de información.
		Tipos de datos objeto; atributos y métodos.
		Tipos de datos colección.
	\paragraph{Sesiones:}
		\todo{completar}
	\paragraph{Instrumentos de calificación:}
		\todo{completar}

