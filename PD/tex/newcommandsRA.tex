\newcommand{\RAUNOa}{RA1.a) Se han identificado los bloques que componen la estructura de un programa informático. }
\newcommand{\RAUNOb}{RA1.b) Se han creado proyectos de desarrollo de aplicaciones. }
\newcommand{\RAUNOc}{RA1.c) Se han utilizado entornos integrados de desarrollo. }
\newcommand{\RAUNOd}{RA1.d) Se han identificado los distintos tipos de variables y la utilidad específica de cada uno. }
\newcommand{\RAUNOe}{RA1.e) Se ha modificado el código de un programa para crear y utilizar variables. }
\newcommand{\RAUNOf}{RA1.f) Se han creado y utilizado constantes y literales. }
\newcommand{\RAUNOg}{RA1.g) Se han clasificado, reconocido y utilizado en expresiones los operadores del lenguaje. }
\newcommand{\RAUNOh}{RA1.h) Se ha comprobado el funcionamiento de las conversiones de tipo explícitas e implícitas. }
\newcommand{\RAUNOi}{RA1.i) Se han introducido comentarios en el código. }
\newcommand{\RADOSa}{RA2.a) Se han identificado los fundamentos de la programación orientada a objetos. }
\newcommand{\RADOSb}{RA2.b) Se han escrito programas simples. }
\newcommand{\RADOSc}{RA2.c) Se han instanciado objetos a partir de clases predefinidas. }
\newcommand{\RADOSd}{RA2.d) Se han utilizado métodos y propiedades de los objetos. }
\newcommand{\RADOSe}{RA2.e) Se han escrito llamadas a métodos estáticos. }
\newcommand{\RADOSf}{RA2.f) Se han utilizado parámetros en la llamada a métodos. }
\newcommand{\RADOSg}{RA2.g) Se han incorporado y utilizado librerías de objetos. }
\newcommand{\RADOSh}{RA2.h) Se han utilizado constructores. }
\newcommand{\RADOSi}{RA2.i) Se ha utilizado el entorno integrado de desarrollo en la creación y compilación de programas simples. }
\newcommand{\RATRESa}{RA3.a) Se ha escrito y probado código que haga uso de estructuras de selección. }
\newcommand{\RATRESb}{RA3.b) Se han utilizado estructuras de repetición. }
\newcommand{\RATRESc}{RA3.c) Se han reconocido las posibilidades de las sentencias de salto. }
\newcommand{\RATRESd}{RA3.d) Se ha escrito código utilizando control de excepciones. }
\newcommand{\RATRESe}{RA3.e) Se han creado programas ejecutables utilizando diferentes estructuras de control. }
\newcommand{\RATRESf}{RA3.f) Se han probado y depurado los programas. }
\newcommand{\RATRESg}{RA3.g) Se ha comentado y documentado el código. }
\newcommand{\RATRESh}{RA3.h) Se han creado excepciones. }
\newcommand{\RATRESi}{RA3.i) Se han utilizado aserciones para la detección y corrección de errores durante la fase de desarrollo. }
\newcommand{\RACUATROa}{RA4.a) Se ha reconocido la sintaxis, estructura y componentes típicos de una clase. }
\newcommand{\RACUATROb}{RA4.b) Se han definido clases. }
\newcommand{\RACUATROc}{RA4.c) Se han definido propiedades y métodos. }
\newcommand{\RACUATROd}{RA4.d) Se han creado constructores. }
\newcommand{\RACUATROe}{RA4.e) Se han desarrollado programas que instancien y utilicen objetos de las clases creadas anteriormente. }
\newcommand{\RACUATROf}{RA4.f) Se han utilizado mecanismos para controlar la visibilidad de las clases y de sus miembros. }
\newcommand{\RACUATROg}{RA4.g) Se han definido y utilizado clases heredadas. }
\newcommand{\RACUATROh}{RA4.h) Se han creado y utilizado métodos estáticos. }
\newcommand{\RACUATROi}{RA4.i) Se han creado y utilizado conjuntos y librerías de clases. }
\newcommand{\RACINCOa}{RA5.a) Se ha utilizado la consola para realizar operaciones de entrada y salida de información. }
\newcommand{\RACINCOb}{RA5.b) Se han aplicado formatos en la visualización de la información. }
\newcommand{\RACINCOc}{RA5.c) Se han reconocido las posibilidades de entrada / salida del lenguaje y las librerías asociadas. }
\newcommand{\RACINCOd}{RA5.d) Se han utilizado ficheros para almacenar y recuperar información. }
\newcommand{\RACINCOe}{RA5.e) Se han creado programas que utilicen diversos métodos de acceso al contenido de los ficheros. }
\newcommand{\RACINCOf}{RA5.f) Se han utilizado las herramientas del entorno de desarrollo para crear interfaces gráficos de usuario simples. }
\newcommand{\RACINCOg}{RA5.g) Se han programado controladores de eventos. }
\newcommand{\RACINCOh}{RA5.h) Se han escrito programas que utilicen interfaces gráficos para la entrada y salida de información. }
\newcommand{\RASEISa}{RA6.a) Se han escrito programas que utilicen matrices (arrays). }
\newcommand{\RASEISb}{RA6.b) Se han reconocido las librerías de clases relacionadas con tipos de datos avanzados. }
\newcommand{\RASEISc}{RA6.c) Se han utilizado listas para almacenar y procesar información. }
\newcommand{\RASEISd}{RA6.d) Se han utilizado iteradores para recorrer los elementos de las listas. }
\newcommand{\RASEISe}{RA6.e) Se han reconocido las características y ventajas de cada una de las colecciones de datos disponibles. }
\newcommand{\RASEISf}{RA6.f) Se han creado clases y métodos genéricos. }
\newcommand{\RASEISg}{RA6.g) Se han utilizado expresiones regulares en la búsqueda de patrones en cadenas de texto. }
\newcommand{\RASEISh}{RA6.h) Se han identificado las clases relacionadas con el tratamiento de documentos escritos en diferentes lenguajes de intercambio de datos. }
\newcommand{\RASEISi}{RA6.i) Se han realizado programas que realicen manipulaciones sobre documentos escritos en diferentes lenguajes de intercambio de datos. }
\newcommand{\RASEISj}{RA6.j) Se han utilizado operaciones agregadas para el manejo de información almacenada en colecciones. }
\newcommand{\RASIETEa}{RA7.a) Se han identificado los conceptos de herencia, superclase y subclase. }
\newcommand{\RASIETEb}{RA7.b) Se han utilizado modificadores para bloquear y forzar la herencia de clases y métodos. }
\newcommand{\RASIETEc}{RA7.c) Se ha reconocido la incidencia de los constructores en la herencia. }
\newcommand{\RASIETEd}{RA7.d) Se han creado clases heredadas que sobrescriben la implementación de métodos de la superclase. }
\newcommand{\RASIETEe}{RA7.e) Se han diseñado y aplicado jerarquías de clases. }
\newcommand{\RASIETEf}{RA7.f) Se han probado y depurado las jerarquías de clases. }
\newcommand{\RASIETEg}{RA7.g) Se han realizado programas que implementen y utilicen jerarquías de clases. }
\newcommand{\RASIETEh}{RA7.h) Se ha comentado y documentado el código. }
\newcommand{\RASIETEi}{RA7.i) Se han identificado y evaluado los escenarios de uso de interfaces. }
\newcommand{\RASIETEj}{RA7.j) Se han identificado y evaluado los escenarios de utilización de la herencia y la composición. }
\newcommand{\RAOHCOa}{RA8.a) Se han identificado las características de las bases de datos orientadas a objetos. }
\newcommand{\RAOHCOb}{RA8.b) Se ha analizado su aplicación en el desarrollo de aplicaciones mediante lenguajes orientados a objetos. }
\newcommand{\RAOHCOc}{RA8.c) Se han instalado sistemas gestores de bases de datos orientados a objetos. }
\newcommand{\RAOHCOd}{RA8.d) Se han clasificado y analizado los distintos métodos soportados por los sistemas gestores para la gestión de la información almacenada. }
\newcommand{\RAOHCOe}{RA8.e) Se han creado bases de datos y las estructuras necesarias para el almacenamiento de objetos. }
\newcommand{\RAOHCOf}{RA8.f) Se han programado aplicaciones que almacenen objetos en las bases de datos creadas. }
\newcommand{\RAOHCOg}{RA8.g) Se han realizado programas para recuperar, actualizar y eliminar objetos de las bases de datos. }
\newcommand{\RAOHCOh}{RA8.h) Se han realizado programas para almacenar y gestionar tipos de datos estructurados, compuestos y relacionados. }
\newcommand{\RANUEVEa}{RA9.a) Se han identificado las características y métodos de acceso a sistemas gestores de bases de datos. }
\newcommand{\RANUEVEb}{RA9.b) Se han programado conexiones con bases de datos. }
\newcommand{\RANUEVEc}{RA9.c) Se ha escrito un código para almacenar información en bases de datos. }
\newcommand{\RANUEVEd}{RA9.d) Se han creado programas para recuperar y mostrar información almacenada en bases de datos. }
\newcommand{\RANUEVEe}{RA9.e) Se han efectuado borrados y modificaciones sobre la información almacenada. }
\newcommand{\RANUEVEf}{RA9.f) Se han creado aplicaciones que muestren la información almacenada en bases de datos. }
\newcommand{\RANUEVEg}{RA9.g) Se han creado aplicaciones para gestionar la información presente en bases de datos. }
