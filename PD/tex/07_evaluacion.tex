%%%%%%%%%%%%%%%%%%%%%%%%%%%%%%%%%%%%%%%
%%%%%%%%%%%%%%%%%%%%%%%%%%%%%%%%%%%%%%%
%%%%%%%%%%%%%%%%%%%%%%%%%%%%%%%%%%%%%%%
%%%%%%%%%%%%%%%%%%%%%%%%%%%%%%%%%%%%%%%
%%%%%%%% INDICACIONES DE LA CAM %%%%%%%
%%%%%%%%%%%%%%%%%%%%%%%%%%%%%%%%%%%%%%%
%%%%%%%%%%%%%%%%%%%%%%%%%%%%%%%%%%%%%%%
%%%%%%%%%%%%%%%%%%%%%%%%%%%%%%%%%%%%%%%
%%%%%%%%%%%%%%%%%%%%%%%%%%%%%%%%%%%%%%%
% Programación del módulo:
%	•	La ponderación de cada resultado de aprendizaje para poder obtener la calificación final del módulo, especificando aquellos resultados de aprendizaje que se adquieren total o parcialmente en la empresa y cómo afecta esto a su ponderación.
%	•	La ponderación de cada criterio de evaluación para obtener la calificación de cada resultado de aprendizaje, teniendo en cuenta que el peso de los criterios de evaluación debe asociarse siempre a la formación en centro, reservando a la parte que se desarrolle en la Formación en Centros de Trabajo (FFE) una ponderación específica dentro del resultado de aprendizaje. Los resultados de aprendizaje impartidos en la empresa serán valorados por el tutor como “superados” o “no superados”, según se haya establecido en el correspondiente Plan de Formación.
%	•	El modo en que se calculará la calificación de cada evaluación, siempre atendiendo a los resultados de aprendizaje (RA) y criterios de evaluación (CE) trabajados durante esa evaluación.
%	•	Los instrumentos de evaluación y calificación que se van a utilizar para evaluar cada CE. Posteriormente, a la hora de calificar, el profesor deberá delimitar: 
%		o	En una prueba: qué pregunta corresponde a qué RA/CE y cuánto puntúa.
%		o	En una rúbrica: qué aspectos se van a observar y cuánto puntúan.
%		o	En una lista de control: qué ítems contiene y cómo se puntuará.

%%%%%%%%%%%%%%%%%%%%%%%%%%%%%%%%%%%%%%%%%%%%
%%%%%%%%%%%%%%%%%%%%%%%%%%%%%%%%%%%%%%%%%%%%
%%%%%%%%%%%%%%%%%%%%%%%%%%%%%%%%%%%%%%%%%%%%
%%%%%%%%%%%%%%%%%%%%%%%%%%%%%%%%%%%%%%%%%%%%
%%%%%%%%%%%%%%%%%%%%%%%%%%%%%%%%%%%%%%%%%%%%
%%%%%%%%%%%%%%%%%%%%%%%%%%%%%%%%%%%%%%%%%%%%

\section{Características de la evaluación}
% Contenido de H.1.


La evaluación del alumnado tendrá un carácter criterial, ateniéndose siempre a los criterios de evaluación vinculados a cada resultado de aprendizaje del módulo. Asimismo, conforme a lo establecido en el artículo 36 de la Orden 893/2022, de 21 de abril, de la Consejería de Educación, Universidades, Ciencia y Portavocía, se garantiza el carácter formativo de dicha evaluación, de modo que contribuya a la mejora continua del proceso de enseñanza-aprendizaje.


La metodología enfocada a la adquisición de competencias requiere un sistema de evaluación continua, ya que este facilita la evaluación formativa y proporciona oportunidades constantes de mejora. Sus principales características son:

\begin{itemize}
  \item Realización periódica de actividades y pruebas evaluables, a lo largo del periodo lectivo, que:
    \begin{itemize}
      \item Favorezcan la asimilación y el desarrollo progresivo de los contenidos y competencias del módulo.
      \item Permitan al alumnado evidenciar, de forma continua, cómo va superando dificultades y adquiriendo las competencias propuestas.
    \end{itemize}

  \item Atención a la diversidad, reconociendo que:
    \begin{itemize}
      \item Cada estudiante cuenta con ritmos de aprendizaje, habilidades y capacidades diferentes.
      \item Se emplearán procedimientos variados de evaluación para ofrecer más ocasiones de comprobar el progreso, corregir errores y mejorar.
      \item Se considera la posibilidad de que el alumnado supere sus dificultades en cualquier momento del proceso.
    \end{itemize}

  \item Requisitos fundamentales para la evaluación continua:
    \begin{itemize}
      \item Asistencia regular a clase.
      \item Registro documentado del desempeño de cada alumno.
      \item Análisis y retroalimentación sistemática:
        \begin{itemize}
          \item Informar de los errores cometidos.
          \item Proporcionar pautas de mejora antes de que el retraso en la adquisición de competencias sea irrecuperable.
        \end{itemize}
      \item Autoevaluación docente y adaptación de la enseñanza cuando sea necesario, para optimizar el proceso formativo.
    \end{itemize}
\end{itemize}


\section{Procedimientos de evaluación y criterios de calificación}
	% Contenido de H.2.

\todo{Los tests de moodle se podrán hacer 3 veces y serán 20 preguntas de un banco de 40. La recuperación de este procedimiento de evaluación será realizarlo una 4ª y última vez.}

\todo{Cada PROCEDIMIENTO DE EVALUACIÓN lleva asociado un INSTRUMENTO DE CALIFICACIÓN (Listado agrupado de todo lo que se va a utilizar (que ya está en la UD))
%
En la Programación se presentará un listado completo y organizado de todos los que se aplicarán (pruebas escritas, rúbricas de proyectos, listas de control, registros de observación, etc.), de manera coherente con la estructura de las unidades de trabajo.}




Con este planteamiento, se garantiza que la evaluación sea un proceso continuo, formativo y transparente, proporcionando al alumnado oportunidades reales de aprendizaje y mejora en consonancia con los principios de calidad e inclusión educativa.


	\section{Proceso de evaluación continua y calificación en la evaluación final ordinaria}
	% Contenido de H.3.

	\section{Proceso de evaluación para alumnos a los que no se puede aplicar la evaluación continua (pérdida del derecho a la evaluación continua)}
	% Contenido de H.4.



	\section{Proceso de evaluación y calificación en la evaluación final extraordinaria}
	% Contenido de H.5.
	\section{Medidas para alumnos con necesidad específica de apoyo educativo (2 ptos)}
	% Contenido de H.6.
	\section{Procedimiento de evaluación para alumnos con el módulo pendiente}

	\subsection{Ordinaria}
	% Detalles de evaluación ordinaria
	\subsection{Extraordinaria}
	% Detalles de evaluación extraordinaria
	\section{Calendario de evaluaciones parciales, final ordinaria y final extraordinaria}
	% Contenido del calendario

\section{Evaluación de la práctica docente}
\todo{Evaluación de la práctica docente}

Se pueden poner encuesta que hacen los alumnos.

Diario de sesiones con mis comentarios de cómo ha ido la sesión.