Se imparte en un instituto público situado en el sureste de Madrid, en un distrito urbano que combina áreas residenciales consolidadas con zonas en expansión y crecimiento demográfico. El tejido socioeconómico local está compuesto principalmente por pequeños comercios familiares, servicios comunitarios, centros educativos, entidades sociales y culturales, así como establecimientos de hostelería y restauración con presencia creciente debido a la ampliación reciente de zonas residenciales, con una limitada presencia de pequeñas empresas tecnológicas.

El centro educativo dispone de instalaciones adecuadas y equipadas tecnológicamente, lo que favorece un entorno propicio para el desarrollo de competencias prácticas y teóricas relacionadas con la informática y las comunicaciones. Además, la ubicación del instituto facilita ciertas oportunidades de colaboración con empresas locales para realizar prácticas formativas, aunque dichas oportunidades son limitadas debido a la ausencia de grandes parques empresariales o grandes centros tecnológicos en la zona. No obstante, la comunicación en transporte público con centros tecnológicos situados en el norte de Madrid es muy buena, lo que permite contar con empresas excepcionales para la realización de la Formación en Centros de Trabajo (de aquí en adelante, FCT).

Se trata de un centro grande, que imparte enseñanzas de Educación Secundaria Obligatoria (ESO), Bachillerato, y Ciclos Formativos de Grado Básico, Medio y Superior. Además de todos los ciclos de la familia profesional de Informática y Comunicaciones, el instituto cuenta con ciclos de otras dos familias profesionales: Imagen Personal (estética) y Transporte y Mantenimiento de Vehículos (automoción). Esta diversidad formativa favorece el intercambio interdisciplinar, el enriquecimiento de la vida académica y la participación activa del alumnado en proyectos colaborativos que trascienden las fronteras de cada especialidad.

El proyecto educativo del centro está firmemente comprometido con la inclusión, la innovación metodológica y la orientación profesional del alumnado. A través de un enfoque centrado en la adquisición de competencias, el centro promueve el uso responsable de la tecnología, la colaboración interdisciplinar y el pensamiento crítico, con el objetivo de formar profesionales técnicos altamente capacitados y socialmente comprometidos.


El grupo destinatario de esta programación didáctica está formado por 29 estudiantes con edades comprendidas mayoritariamente entre los 18 y 22 años, con algún alumno rondando los 30 años. Este grupo destaca por la heterogeneidad de su alumnado, incluyendo estudiantes procedentes tanto del Bachillerato como del ámbito laboral o de otros ciclos formativos, fundamentalmente de Sistemas Microinformáticos y Redes (SMR). Dentro del grupo, se han identificado necesidades educativas especiales (NEE) en tres estudiantes: uno de ellos presenta dislexia, lo que afecta a su velocidad y precisión lectora, así como a la comprensión de textos escritos; otro presenta un Trastorno por Déficit de Atención (TDA) sin hiperactividad, que se manifiesta en dificultades para mantener la concentración de forma sostenida; y el tercero muestra rasgos compatibles con un Trastorno del Espectro Autista (TEA) de nivel 1, con especial sensibilidad a los cambios de rutina y dificultades en la interacción social espontánea, lo que implica la adopción de estrategias metodológicas flexibles, atención individualizada y medidas específicas de adaptación curricular, asegurando así una adecuada inclusión educativa y el éxito académico de todos los estudiantes. En este sentido, se aplicarán los principios del Diseño Universal para el Aprendizaje (DUA), que permiten flexibilizar los métodos, materiales y formas de participación y evaluación, facilitando así el acceso equitativo a los contenidos del módulo para todo el alumnado.







