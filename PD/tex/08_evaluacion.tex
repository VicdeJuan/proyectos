
- La evaluación de los alumnos será CRITERIAL: se realizará según los criterios de evaluación establecidos para los resultados de aprendizaje del módulo.
También hay que tener en cuenta el artículo 36 de la Orden 893/2022, de 21 de abril, de la Consejería de Educación, Universidades, Ciencia y Portavocía, que recoge el carácter FORMATIVO de la evaluación

La metodología dirigida a la adquisición de competencias exige un método de evaluación continua, ya que favorece la evaluación formativa. Las características de la evaluación continua son:
•	Realización periódica de actividades y pruebas evaluables durante el periodo lectivo que permitan:
o	La asimilación y el desarrollo progresivo de los contenidos y competencias del módulo.
o	Proporcionar al alumnado, a lo largo de todo el periodo, situaciones en las que demostrar que va superando sus dificultades y adquiriendo las competencias.
•	Atención a la diversidad de los estudiantes, reconociendo que no todos tendrán las mismas habilidades ni capacidades, y que aprenderán a ritmos diferentes. Esto implica que:
o	Se evalúa con diferentes procedimientos.
o	Se proporcionan más ocasiones para comprobar el proceso de aprendizaje, la disminución del error y la mejora.
o	Se contempla la posibilidad de que el alumno logre superar sus dificultades hasta el final.
•	Requisitos de la evaluación continua:
o	La asistencia a clase del alumno.
o	Llevar un registro del desempeño del alumno.
o	Valorar el proceso de aprendizaje del alumnado: 
- 	Informarle de sus errores.
-	Dar pautas para la mejora antes de que se retrasen demasiado.
o	Valorar el proceso de enseñanza del profesorado e introducir los cambios necesarios para mejorarlo.
Por ello, es importante que se incluya en la Programación del módulo:
•	La ponderación de cada resultado de aprendizaje para poder obtener la calificación final del módulo, especificando aquellos resultados de aprendizaje que se adquieren total o parcialmente en la empresa y cómo afecta esto a su ponderación.
•	La ponderación de cada criterio de evaluación para obtener la calificación de cada resultado de aprendizaje, teniendo en cuenta que el peso de los criterios de evaluación debe asociarse siempre a la formación en centro, reservando a la parte que se desarrolle en la Formación en Centros de Trabajo (FFE) una ponderación específica dentro del resultado de aprendizaje. Los resultados de aprendizaje impartidos en la empresa serán valorados por el tutor como “superados” o “no superados”, según se haya establecido en el correspondiente Plan de Formación.
•	El modo en que se calculará la calificación de cada evaluación, siempre atendiendo a los resultados de aprendizaje (RA) y criterios de evaluación (CE) trabajados durante esa evaluación.
•	Los instrumentos de evaluación y calificación que se van a utilizar para evaluar cada CE. Posteriormente, a la hora de calificar, el profesor deberá delimitar: 
o	En una prueba: qué pregunta corresponde a qué RA/CE y cuánto puntúa.
o	En una rúbrica: qué aspectos se van a observar y cuánto puntúan.
o	En una lista de control: qué ítems contiene y cómo se puntuará.


\section{Características de la evaluación}
% Contenido de H.1.
\section{Procedimientos de evaluación y criterios de calificación}
% Contenido de H.2.

Cada PROCEDIMIENTO DE EVALUACIÓN lleva asociado un INSTRUMENTO DE CALIFICACIÓN
Listado agrupado de todo lo que se va a utilizar (que ya está en la UD)


\section{Proceso de evaluación continua y calificación en la evaluación final ordinaria}
% Contenido de H.3.
\section{Proceso de evaluación para alumnos a los que no se puede aplicar la evaluación continua (pérdida del derecho a la evaluación continua)}
% Contenido de H.4.
\section{Proceso de evaluación y calificación en la evaluación final extraordinaria}
% Contenido de H.5.
\section{Medidas para alumnos con necesidad específica de apoyo educativo (2 ptos)}
% Contenido de H.6.
\section{Procedimiento de evaluación para alumnos con el módulo pendiente}

\subsection{Ordinaria}
% Detalles de evaluación ordinaria
\subsection{Extraordinaria}
% Detalles de evaluación extraordinaria
\section{Calendario de evaluaciones parciales, final ordinaria y final extraordinaria}
% Contenido del calendario
