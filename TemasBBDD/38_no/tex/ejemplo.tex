\documentclass[12pt]{report}

\usepackage[utf8]{inputenc}
\usepackage[spanish]{babel}
\usepackage{lmodern}
\usepackage[T1]{fontenc}
\usepackage{geometry}
\usepackage{hyperref}
\usepackage{setspace}
\usepackage{amsmath, amssymb}
\usepackage{graphicx}
\geometry{a4paper, margin=1in}


\usepackage{tikz}
\usetikzlibrary{automata, positioning} % Librerías necesarias

\onehalfspacing

\title{\textbf{Tema 38 – Modelo de datos relacional. Estructuras. Operaciones. Álgebra relacional }}    
\author{}
\date{\today}

\begin{document}

\begin{figure}[htbp]
    \centering
    \begin{tikzpicture}[
        ->,
        >=stealth,
        shorten >=1pt,
        auto,
        node distance=2.8cm,
        semithick
        ]

        \node[state] (rel1) {Relación R1};
        \node[state] (rel2) [right of=rel1] {Relación R2};
        \node[state] (union) [below of=rel1] {Unión ($\cup$)};
        \node[state] (select) [below of=rel2] {Selección ($\sigma$)};
        \node[state] (join) [right of=union] {Natural Join ($\bowtie$)};
        \node[state] (project) [below of=join] {Proyección ($\Pi$)};
    
        \path (rel1) edge (union)
              (rel2) edge (union)
              (rel1) edge (select)
              (select) edge (project)
              (rel1) edge (join)
              (rel2) edge (join)
              (union) edge (join);
    
        \draw[dashed] (rel1) -- (0,-4.2);
        \draw[dashed] (rel2) -- (6,-4.2);
        \draw[dashed] (join) -- (4,-5.3);
    
    \end{tikzpicture}
    \caption{Ejemplo de composición de operaciones del álgebra relacional.}
    \label{fig:algebra}
\end{figure}


\end{document}
