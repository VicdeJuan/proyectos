\section{Introducción a los Modelos de Datos}

\noindent En la base de cualquier sistema de bases de datos robusto se encuentra el \textbf{modelo de datos}, definido como una \textbf{colección de herramientas conceptuales para describir los datos, sus relaciones, su semántica y las restricciones de consistencia}. El rol primordial de un modelo de datos es ofrecer un \textbf{modo de describir el diseño de las bases de datos en los niveles físico, lógico y de vistas}. En esencia, un modelo de datos proporciona la abstracción necesaria para que diferentes usuarios y el propio sistema gestor de bases de datos (DBMS) puedan interactuar con los datos a un nivel adecuado a sus necesidades, sin tener que preocuparse por los intrincados detalles de la implementación física.

\noindent Los modelos de datos actúan como un \textbf{tipo de abstracción de datos} que se utiliza para proporcionar una \textbf{representación conceptual de los datos}. Esta representación emplea \textbf{conceptos lógicos, como objetos, sus propiedades y sus relaciones}, que resultan más intuitivos para la mayoría de los usuarios que los detalles del almacenamiento físico en el computador. Al \textbf{ocultar los detalles del almacenamiento y de la implementación} que no son relevantes para la mayoría de los usuarios, los modelos de datos simplifican la interacción con la base de datos y facilitan el diseño y la comprensión de su estructura.

\noindent Es importante distinguir entre la \textbf{descripción de la base de datos}, denominada \textbf{esquema de la base de datos}, y la \textbf{base de datos misma}, que es el \textbf{estado de la base de datos} en un momento dado. El esquema se especifica durante la fase de diseño y se espera que cambie con poca frecuencia, mientras que el estado de la base de datos evoluciona constantemente a medida que se insertan, eliminan o modifican los datos. Los modelos de datos proporcionan las convenciones para definir estos esquemas y para comprender la estructura subyacente de los datos.

\noindent Se han propuesto numerosos modelos de datos a lo largo de la historia de las bases de datos, y estos se pueden \textbf{clasificar conforme a los tipos de conceptos que utilizan para describir la estructura de la base de datos}. Una clasificación común distingue entre tres categorías principales:

\begin{itemize}
    \item \textbf{Modelos de datos de alto nivel o conceptuales:} Estos modelos ofrecen \textbf{conceptos muy cercanos a como muchos usuarios perciben los datos}. Utilizan conceptos como \textbf{entidades, atributos y relaciones}. Una \textbf{entidad representa un objeto o concepto del mundo real} de interés para la base de datos, como un empleado o un proyecto. Un \textbf{atributo representa alguna propiedad de interés que describe a una entidad}, como el nombre o el salario de un empleado. Una \textbf{relación entre dos o más entidades representa una asociación} entre ellas, como una relación de trabajo entre un empleado y un proyecto. El \textbf{modelo Entidad-Relación (ER)} y su versión mejorada, el \textbf{modelo EER}, son ejemplos prominentes de modelos de datos conceptuales de alto nivel, frecuentemente utilizados como un primer paso en el diseño de los esquemas de las bases de datos. Estos modelos permiten a los diseñadores \textbf{identificar las entidades que se van a representar en la base de datos y el modo en que se relacionan entre sí}.

    \begin{itemize}
    \item \textit{Profundizando en los Modelos ER y EER:} Para los ingenieros informáticos, es crucial comprender las fortalezas y limitaciones de los modelos ER y EER. Si bien el modelo ER proporciona una representación inicial clara y concisa, el modelo EER introduce conceptos más avanzados como generalización/especialización (herencia), agregación y categorías, lo que permite modelar escenarios del mundo real más complejos. Por ejemplo, en un sistema de gestión de recursos humanos, podríamos tener una entidad "Empleado" con atributos como nombre, ID y salario. Mediante generalización/especialización, podríamos crear subentidades como "EmpleadoPorHoras" y "EmpleadoSalariado," cada una con atributos adicionales específicos (tarifa por hora y salario anual, respectivamente). La agregación nos permitiría modelar la relación entre "Empleado," "Proyecto" y "Tarea" como una única unidad conceptual.  La elección entre ER y EER dependerá de la complejidad del dominio del problema y del nivel de detalle necesario en la representación. Es importante considerar las implicaciones en la mantenibilidad y la escalabilidad del esquema al optar por un modelo más complejo.  Además,  los ingenieros deben ser conscientes de las diferentes notaciones gráficas (Chen, UML) que se utilizan para representar diagramas ER y EER, y ser capaces de interpretar y utilizar cualquiera de ellas de forma efectiva.
    \end{itemize}

    \item \textbf{Modelos de datos de bajo nivel o físicos:} Estos modelos ofrecen \textbf{conceptos que describen los detalles de cómo se almacenan los datos en el computador}. Los conceptos proporcionados por los modelos de datos de bajo nivel están pensados principalmente para los \textbf{especialistas en computadores}, no para los usuarios finales. Describen la \textbf{información como formatos de registro, ordenación de registros y rutas de acceso}. Una \textbf{ruta de acceso es una estructura que hace más eficaz la búsqueda de registros en una base de datos}. El \textbf{nivel físico de abstracción describe cómo se almacenan realmente los datos}, detallando las \textbf{estructuras de datos complejas de bajo nivel}. Los capítulos dedicados al almacenamiento físico y las estructuras de acceso profundizan en estos modelos.

    \begin{itemize}
    \item \textit{Detalles Críticos para Ingenieros:}  En este nivel, los ingenieros deben comprender cómo las decisiones de diseño de la base de datos impactan directamente en el rendimiento del sistema.  Conocimientos sobre estructuras de datos como B-Trees, hashing, y diferentes tipos de índices (densos, dispersos) son esenciales.  Además, la comprensión de cómo el sistema operativo interactúa con el DBMS para la gestión de memoria y el almacenamiento en disco es crucial. Por ejemplo, entender cómo la fragmentación del disco afecta la velocidad de acceso a los datos, o cómo configurar correctamente el buffer pool del DBMS para maximizar el uso de la memoria RAM.  También es fundamental comprender los diferentes tipos de RAID (Redundant Array of Independent Disks) y su impacto en la disponibilidad y rendimiento de la base de datos.  Finalmente, el conocimiento de los diferentes formatos de almacenamiento de datos (row-oriented vs. column-oriented) y sus implicaciones en las cargas de trabajo OLTP (Online Transaction Processing) y OLAP (Online Analytical Processing) es fundamental para tomar decisiones informadas sobre el diseño físico de la base de datos.
    \end{itemize}

    \item \textbf{Modelos de datos representativos (o de implementación):} Situados entre los modelos conceptuales y físicos, estos modelos ofrecen \textbf{conceptos que los usuarios finales pueden entender pero que no están demasiado alejados de cómo se organizan los datos dentro del computador}. \textbf{Ocultan algunos detalles relativos al almacenamiento de los datos, pero pueden implementarse directamente en un computador}. Los modelos de datos representativos son los \textbf{más utilizados en los DBMS comerciales tradicionales}. El \textbf{modelo de datos relacional}, que organiza los datos en \textbf{un conjunto de tablas} donde cada tabla representa una relación entre un conjunto de valores, es el ejemplo más ampliamente utilizado. Los \textbf{modelos de datos heredados (red y jerárquico)} y la familia de \textbf{modelos de datos de objetos (ODMG)} también se incluyen en esta categoría. Estos modelos a menudo se conocen como \textbf{modelos de datos basados en registros}.

    \begin{itemize}
    \item \textit{El Modelo Relacional en Profundidad:} El modelo relacional, basado en la teoría de conjuntos y la lógica de predicados, es la base de la mayoría de los DBMS comerciales. Los ingenieros informáticos deben dominar los conceptos de relaciones (tablas), atributos (columnas), tuplas (filas), claves primarias y foráneas, y la integridad referencial.  La normalización es un proceso crucial para minimizar la redundancia y garantizar la integridad de los datos, y los ingenieros deben comprender las diferentes formas normales (1NF, 2NF, 3NF, BCNF, 4NF, 5NF) y cómo aplicarlas de manera efectiva.  El lenguaje SQL (Structured Query Language) es el estándar para interactuar con bases de datos relacionales, y los ingenieros deben ser competentes en la escritura de consultas SQL complejas, incluyendo joins, subconsultas, agregaciones y funciones de ventana.  Además, es importante comprender cómo los optimizadores de consultas de los DBMS relacionales funcionan para mejorar el rendimiento de las consultas, y cómo los índices pueden ser utilizados para acelerar las búsquedas de datos.  Finalmente, la comprensión de los principios de ACID (Atomicity, Consistency, Isolation, Durability) es fundamental para garantizar la confiabilidad y la integridad de las transacciones en un sistema de base de datos relacional.
    \end{itemize}
\end{itemize}

\noindent La elección de un modelo de datos específico en las primeras fases del diseño es crucial, ya que influye en cómo se capturan y representan los requisitos de los usuarios y cómo se traduce el diseño conceptual en una implementación física eficiente.
