\section{Diccionarios de Datos Activos vs. Pasivos}

Tradicionalmente, se ha distinguido entre dos tipos principales de diccionarios de datos: \textbf{activos} y \textbf{pasivos}. Esta distinción se basa principalmente en cómo se gestionan y mantienen, y en su grado de integración con el DBMS.  La elección entre un enfoque activo o pasivo impacta significativamente la integridad, eficiencia y la capacidad de gestión de la base de datos, representando decisiones de diseño críticas.

\subsection{Diccionarios de Datos Activos}

Un \textbf{diccionario de datos activo} es aquel que está \textbf{directamente integrado con el DBMS y es gestionado y actualizado automáticamente por el propio sistema}. Cuando se realiza cualquier cambio en la estructura de la base de datos a través del Lenguaje de Definición de Datos (LDD), como la creación o modificación de tablas, la alteración de atributos o la definición de restricciones, \textbf{el DBMS actualiza inmediatamente los metadatos correspondientes en el diccionario de datos}. Esto asegura que el diccionario de datos \textbf{siempre refleje con precisión la estructura actual de la base de datos}. La mayoría de los DBMSs modernos utilizan un diccionario de datos activo.  En esencia, se trata de un repositorio de metadatos que es una parte integral del motor de la base de datos, actuando como un autómata que se mantiene al día con cada alteración estructural.

\subsubsection{Ventajas de los Diccionarios de Datos Activos}

\begin{itemize}
    \item \textbf{Consistencia:} La principal ventaja es la \textbf{garantía de que el diccionario de datos siempre está sincronizado con la estructura real de la base de datos}. Esto elimina el riesgo de inconsistencias entre los metadatos y los datos, lo que podría llevar a errores en el procesamiento de consultas o en la aplicación de restricciones. La consistencia se erige como la piedra angular de la fiabilidad del sistema.  Un diccionario activo elimina el riesgo de derivas causadas por actualizaciones manuales olvidadas o incorrectas, mitigando el riesgo de errores críticos en el procesamiento de datos y la aplicación de reglas de negocio.
    \item \textbf{Integración:} Al estar integrado con el DBMS, el diccionario de datos \textbf{es accesible directamente por todos los componentes del sistema}, como el procesador de consultas, los módulos de seguridad y las utilidades de administración. Esto facilita la operación y el mantenimiento del sistema.  La integración permite que el optimizador de consultas, por ejemplo, tome decisiones más informadas basadas en información precisa sobre el tamaño de las tablas, los tipos de datos y las restricciones.  Los módulos de seguridad pueden usar el diccionario para determinar los privilegios de acceso de los usuarios. Las utilidades de administración pueden consultar el diccionario para obtener información sobre el esquema de la base de datos.
    \item \textbf{Automatización:} La \textbf{actualización automática de los metadatos} reduce la carga administrativa y elimina la posibilidad de errores humanos que podrían ocurrir en la gestión manual de un diccionario pasivo. La automatización minimiza el coste operacional y reduce la ventana de oportunidad para errores humanos, dos factores críticos en entornos de bases de datos de gran escala.
\end{itemize}

\subsubsection{Desventajas de los Diccionarios de Datos Activos}

\begin{itemize}
    \item \textbf{Sobrecarga:} La actualización automática del diccionario de datos cada vez que se modifica el esquema puede \textbf{introducir una pequeña sobrecarga de rendimiento}, especialmente en sistemas con una alta frecuencia de cambios en el esquema. Sin embargo, esta sobrecarga suele ser mínima en comparación con los beneficios de la consistencia. Esta sobrecarga, aunque generalmente insignificante, puede ser más pronunciada en sistemas OLTP con una alta tasa de transacciones y modificaciones de esquema frecuentes.  En estos casos, las estrategias de indexación y caching del diccionario deben ser cuidadosamente optimizadas.
\end{itemize}

\subsection{Diccionarios de Datos Pasivos}

Por otro lado, un \textbf{diccionario de datos pasivo} es un \textbf{sistema separado del DBMS} que almacena información sobre la base de datos y otros sistemas de información. La \textbf{gestión y actualización de un diccionario de datos pasivo se realiza de forma independiente}, a menudo por personal especializado (administradores del diccionario de datos) utilizando herramientas específicas. Los cambios realizados en la estructura de la base de datos \textbf{no se reflejan automáticamente en el diccionario pasivo}; en cambio, es necesario realizar actualizaciones manuales para mantenerlo sincronizado. En contraste con el enfoque activo, el diccionario pasivo se concibe como un artefacto externo al DBMS, requiriendo una gestión manual que puede ser propensa a errores y demoras.

\subsubsection{Ventajas de los Diccionarios de Datos Pasivos (históricamente relevantes, pero menos comunes hoy en día para la gestión exclusiva de bases de datos)}

\begin{itemize}
    \item \textbf{Independencia:} Al ser un sistema separado, un diccionario pasivo puede \textbf{almacenar información sobre múltiples bases de datos y otros sistemas de información}, proporcionando una visión más amplia de los metadatos de una organización. Esta independencia puede ser útil en entornos empresariales complejos donde es necesario mantener una visión holística de los metadatos a través de múltiples sistemas heterogéneos.
    \item \textbf{Control:} La gestión independiente permite un \textbf{mayor control sobre la información que se almacena} en el diccionario y cómo se estructura, lo que puede ser útil para fines de documentación y planificación. El control granular sobre la información almacenada puede ser esencial para cumplir con requisitos regulatorios específicos o para adaptar el diccionario a las necesidades particulares de la organización.
\end{itemize}

\subsubsection{Desventajas de los Diccionarios de Datos Pasivos}

\begin{itemize}
    \item \textbf{Inconsistencia:} La principal desventaja es la \textbf{posibilidad de que el diccionario de datos se desincronice con la estructura real de la base de datos} si no se realizan las actualizaciones manuales de manera oportuna y precisa. Esta inconsistencia puede llevar a documentación obsoleta y a la toma de decisiones erróneas basadas en información desactualizada.  La inconsistencia representa un riesgo significativo, ya que puede conducir a decisiones de diseño basadas en información incorrecta, optimizaciones ineficientes de consultas y, en última instancia, a errores en el procesamiento de datos.
    \item \textbf{Falta de integración:} Al no estar directamente integrado con el DBMS, el diccionario pasivo \textbf{no puede ser utilizado directamente por el procesador de consultas u otros componentes del sistema} para validar o optimizar las operaciones.  La falta de integración impide el uso del diccionario para la optimización automática de consultas, la validación de la integridad de los datos en tiempo real y otras funcionalidades que mejorarían significativamente el rendimiento y la fiabilidad del sistema.
    \item \textbf{Mayor carga administrativa:} La \textbf{necesidad de realizar actualizaciones manuales} implica una mayor carga de trabajo y aumenta el riesgo de errores humanos. La carga administrativa asociada con la gestión manual de un diccionario pasivo puede ser significativa, especialmente en entornos con frecuentes cambios en el esquema de la base de datos. Esto requiere personal especializado y un proceso riguroso de gestión de cambios para minimizar el riesgo de inconsistencias.
\end{itemize}

\subsection{Tendencias Modernas y Convergencia}

En la práctica moderna, la mayoría de los DBMSs comerciales implementan \textbf{diccionarios de datos activos} debido a la importancia de la consistencia y la integración para el funcionamiento eficiente y confiable de la base de datos. Los diccionarios de datos pasivos todavía pueden utilizarse en algunas organizaciones para la gestión de metadatos a nivel empresarial, abarcando múltiples sistemas, pero incluso en estos casos, a menudo se busca la manera de automatizar la sincronización con los metadatos gestionados activamente por los DBMSs individuales. La tendencia actual apunta a una convergencia entre los enfoques activo y pasivo.  Las organizaciones buscan mantener la consistencia y la integración que ofrecen los diccionarios activos dentro de cada DBMS, al tiempo que extienden la visión de los metadatos a nivel empresarial a través de herramientas que automatizan la sincronización entre los diccionarios activos de cada sistema y un repositorio centralizado de metadatos.  Este repositorio centralizado, aunque gestionado de forma pasiva, se mantiene actualizado mediante la extracción y transformación de los metadatos de los diccionarios activos de cada DBMS.  Esta estrategia híbrida permite una visión completa de los metadatos a nivel empresarial al tiempo que minimiza el riesgo de inconsistencias.
