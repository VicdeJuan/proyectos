\documentclass[a4paper,12pt]{article}
\usepackage[utf8]{inputenc}    
\usepackage[spanish]{babel}    
\usepackage{tikz}              
\usetikzlibrary{arrows.meta, positioning, shapes.geometric}
\usepackage{float}             
\usepackage{amsmath}
\usepackage{hyperref}
\usepackage{geometry}
\geometry{margin=1in}

\title{Ejercicios de Direccionamiento IP y VLSM}
\author{Tu Nombre}
\date{\today}

\begin{document}
\maketitle

\section{Ejercicio 1: Red Empresarial Distribuida con Redundancia}

\subsection{Enunciado}
Utilizando la red \textbf{192.168.0.0/16}, realiza el direccionamiento IP mediante VLSM para la siguiente topología:
\begin{itemize}
  \item \textbf{Routers:} R1, R2, R3 y R4.
  \item \textbf{Interconexiones entre routers:}  
    \begin{itemize}
      \item R1--R2, R2--R3, R3--R4 y R4--R1 (formando un anillo).
      \item Enlace adicional entre R1 y R3 para mayor redundancia.
    \end{itemize}
  \item \textbf{Redes LAN locales:}
    \begin{itemize}
      \item R1: Red A con 400 PCs.
      \item R2: Red B con 250 PCs.
      \item R3: Red C con 600 PCs.
      \item R4: Red D con 150 PCs.
    \end{itemize}
  \item \textbf{Enlaces punto a punto:} Todos los enlaces entre routers se deben asignar con bloques \texttt{}.
\end{itemize}

\subsection{Diagrama de Topología con TikZ}
\begin{figure}[H]
  \centering
  \begin{tikzpicture}[
      node distance=2.5cm and 2.5cm,
      router/.style={draw, circle, minimum size=1cm, fill=blue!20},
      lan/.style={draw, rectangle, minimum width=2.5cm, minimum height=0.8cm, fill=green!20},
      link/.style={-Stealth, thick},
      auto
    ]
    % Posiciones de los routers (dispuestos en forma de cuadrado)
    \node[router] (R1) {R1};
    \node[router, right=of R1] (R2) {R2};
    \node[router, below=of R2] (R3) {R3};
    \node[router, below=of R1] (R4) {R4};
    
    % Enlaces entre routers (anillo)
    \draw[link] (R1) -- node[above] {} (R2);
    \draw[link] (R2) -- node[right] {} (R3);
    \draw[link] (R3) -- node[below] {} (R4);
    \draw[link] (R4) -- node[left] {} (R1);
    
    % Enlace adicional entre R1 y R3
    \draw[link, dashed] (R1) -- node[above, sloped] {} (R3);
    
    % LANs conectadas a cada router
    \node[lan, above=of R1] (LAN_A) {Red A\\ (400 PCs)};
    \draw[link] (R1) -- (LAN_A);
    
    \node[lan, above=of R2] (LAN_B) {Red B\\ (250 PCs)};
    \draw[link] (R2) -- (LAN_B);
    
    \node[lan, below=of R3] (LAN_C) {Red C\\ (600 PCs)};
    \draw[link] (R3) -- (LAN_C);
    
    \node[lan, below=of R4] (LAN_D) {Red D\\ (150 PCs)};
    \draw[link] (R4) -- (LAN_D);
    
  \end{tikzpicture}
  \caption{Topología de Red Empresarial Distribuida con Redundancia.}
\end{figure}

\newpage

\section{Ejercicio 2: Campus Universitario con Topología Mesh Parcial}

\subsection{Enunciado}
Utilizando la red \textbf{172.20.0.0/16}, realiza el direccionamiento IP mediante VLSM para la siguiente topología de campus:
\begin{itemize}
  \item \textbf{Routers:} R1, R2, R3, R4 y R5.
  \item \textbf{Interconexiones entre routers:}
    \begin{itemize}
      \item R1 se conecta con R2 y R3.
      \item R2 se conecta con R3 y R4.
      \item R3 se conecta con R4 y R5.
      \item R4 se conecta con R5.
    \end{itemize}
    (Todos los enlaces se implementan con bloques \texttt{}).
  \item \textbf{Redes LAN locales:}
    \begin{itemize}
      \item R1: Área de Docencia con 350 equipos.
      \item R2: Área Administrativa con 200 equipos.
      \item R3: Laboratorio de Investigación con 450 equipos.
      \item R4: Biblioteca y Servicios con 150 equipos.
      \item R5: Área de Estudiantes con 500 equipos.
    \end{itemize}
\end{itemize}

\subsection{Diagrama de Topología con TikZ}
\begin{figure}[H]
  \centering
  \begin{tikzpicture}[
      node distance=2cm and 2cm,
      router/.style={draw, circle, minimum size=1cm, fill=blue!20},
      lan/.style={draw, rectangle, minimum width=2.5cm, minimum height=0.8cm, fill=green!20},
      link/.style={-Stealth, thick},
      auto
    ]
    % Posicionamiento de los routers en forma de pentágono
    \node[router] (R1) {R1};
    \node[router, right=of R1] (R2) {R2};
    \node[router, below right=of R2] (R3) {R3};
    \node[router, below left=of R3] (R4) {R4};
    \node[router, left=of R4] (R5) {R5};
    
    % Conexiones entre routers
    \draw[link] (R1) -- node[above] {} (R2);
    \draw[link] (R1) -- node[above left] {} (R3);
    \draw[link] (R2) -- node[right] {} (R3);
    \draw[link] (R2) -- node[below right] {} (R4);
    \draw[link] (R3) -- node[below] {} (R4);
    \draw[link] (R3) -- node[below right] {} (R5);
    \draw[link] (R4) -- node[below left] {} (R5);
    
    % LANs conectadas a cada router
    \node[lan, above=of R1] (LAN1) {Docencia (350 equipos)};
    \draw[link] (R1) -- (LAN1);
    
    \node[lan, right=of R2] (LAN2) {Administrativa (200 equipos)};
    \draw[link] (R2) -- (LAN2);
    
    \node[lan, right=of R3] (LAN3) {Laboratorio (450 equipos)};
    \draw[link] (R3) -- (LAN3);
    
    \node[lan, below=of R4] (LAN4) {Biblioteca y Servicios (150 equipos)};
    \draw[link] (R4) -- (LAN4);
    
    \node[lan, left=of R5] (LAN5) {Estudiantes (500 equipos)};
    \draw[link] (R5) -- (LAN5);
    
  \end{tikzpicture}
  \caption{Topología de Campus Universitario con Topología Mesh Parcial.}
\end{figure}

\newpage

\section{Ejercicio 3: Red Regional con Topología Mixta y Enlaces Cruzados}

\subsection{Enunciado}
Con la red \textbf{10.0.0.0/16}, realiza el direccionamiento mediante VLSM para la siguiente topología regional:
\begin{itemize}
  \item \textbf{Routers:} R1, R2, R3 y R4.
  \item \textbf{Interconexiones entre routers:}
    \begin{itemize}
      \item R1 se conecta a R2 y R3.
      \item R2 se conecta a R3 y R4.
      \item R3 se conecta a R4.
      \item Existe un enlace cruzado entre R1 y R4 para soportar tráfico alternativo.
    \end{itemize}
    (Todos estos enlaces son \texttt{}).
  \item \textbf{Redes LAN locales:}
    \begin{itemize}
      \item R1: Oficina Central con 500 PCs.
      \item R2: Sede Norte con 300 PCs.
      \item R3: Sede Sur con 400 PCs.
      \item R4: Sede Este con 250 PCs.
    \end{itemize}
\end{itemize}

\subsection{Diagrama de Topología con TikZ}
\begin{figure}[H]
  \centering
  \begin{tikzpicture}[
      node distance=2.5cm and 2.5cm,
      router/.style={draw, circle, minimum size=1cm, fill=blue!20},
      lan/.style={draw, rectangle, minimum width=2.5cm, minimum height=0.8cm, fill=green!20},
      link/.style={-Stealth, thick},
      auto
    ]
    % Posiciones de los routers en forma de cuadrado
    \node[router] (R1) {R1};
    \node[router, right=of R1] (R2) {R2};
    \node[router, below=of R2] (R3) {R3};
    \node[router, below=of R1] (R4) {R4};
    
    % Conexiones principales
    \draw[link] (R1) -- node[above] {} (R2);
    \draw[link] (R1) -- node[left] {} (R3);
    \draw[link] (R2) -- node[right] {} (R3);
    \draw[link] (R2) -- node[above right] {} (R4);
    \draw[link] (R3) -- node[below right] {} (R4);
    % Enlace cruzado entre R1 y R4
    \draw[link, dashed] (R1) -- node[below left, sloped] {} (R4);
    
    % LANs conectadas a cada router
    \node[lan, above=of R1] (LAN_R1) {Oficina Central (500 PCs)};
    \draw[link] (R1) -- (LAN_R1);
    
    \node[lan, right=of R2] (LAN_R2) {Sede Norte (300 PCs)};
    \draw[link] (R2) -- (LAN_R2);
    
    \node[lan, right=of R3] (LAN_R3) {Sede Sur (400 PCs)};
    \draw[link] (R3) -- (LAN_R3);
    
    \node[lan, left=of R4] (LAN_R4) {Sede Este (250 PCs)};
    \draw[link] (R4) -- (LAN_R4);
    
  \end{tikzpicture}
  \caption{Topología de Red Regional con Topología Mixta y Enlaces Cruzados.}
\end{figure}

\section{Conclusión}
Cada uno de estos ejercicios requiere que se determine el tamaño mínimo de cada subred (incluyendo red y broadcast), se calcule el prefijo adecuado y se asignen los bloques IP mediante VLSM, completando la tabla de subredes con la información necesaria. Estos ejemplos permiten practicar la segmentación de redes en topologías complejas y no jerárquicas.

\end{document}
