
\documentclass[a4paper,12pt]{article}
\usepackage[utf8]{inputenc}
\usepackage[spanish]{babel}
\usepackage{geometry}
\geometry{margin=2.5cm}
\title{Ejercicios de Codificación Digital}
\author{}
\date{}
\begin{document}
\maketitle
\section*{Ejercicio 1}
Dada la siguiente secuencia binaria: [1, 0, 1, 1, 0], represéntala utilizando la codificación NRZ.

\section*{Ejercicio 2}
Dada la secuencia [0, 1, 0, 0, 1, 1], dibuja la señal correspondiente con codificación Manchester.


\section*{Ejercicio 4}
Dibuja la señal que resulta de codificar la secuencia [1, 0, 0, 1] utilizando codificación RZ.

\section*{Ejercicio 5}
¿Cuál es la principal ventaja de la codificación Manchester frente a NRZ?


\section*{Ejercicio 7}
Dibuja la señal de la secuencia [0, 1, 1, 0, 1] usando Manchester Diferencial.

\section*{Ejercicio 8}
En una señal codificada, observas una alternancia clara entre voltajes positivos y negativos para los '1', y niveles planos para los '0'. ¿Qué tipo de codificación es más probable?

\section*{Ejercicio 9}
Dibuja la señal correspondiente a la secuencia binaria [1, 0, 1, 0, 0, 1, 1] utilizando codificación NRZI.

\section*{Ejercicio 10}
Observa una señal codificada con transiciones en el centro de cada bit. ¿Qué codificación podría ser?

\section*{Ejercicio 11}
¿Qué codificación resulta más robusta ante errores de sincronización: NRZ o Manchester Diferencial?

\section*{Ejercicio 13}
En una señal codificada no hay secuencias de más de tres ceros consecutivos. ¿Qué tipo de codificación podría ser?


\section*{Ejercicio 15}
Una señal codificada muestra una alternancia constante entre 1 y -1 en los bits '1', mientras los '0' se ven planos. ¿Qué codificación es?

\section*{Ejercicio 16}
¿Qué ventaja tiene el uso de codificación RZ frente a NRZ?

\end{document}